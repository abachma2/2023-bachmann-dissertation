\section{Methodology}
Optimization of the nuclear fuel cycle is intended to develop 
a fuel cycle based on a specific objective or objectives. For this 
work, we developed three different optimization problems to apply to 
Scenarios 7 (no growth, once-through transition to the \gls{MMR}, Xe-100, 
and VOYGR) and 14 (no growth, limited recycling transition to the 
\gls{MMR}, Xe-100, and VOYGR): minimizing the \gls{SWU} capacity to 
produce \gls{HALEU}, minimizing the mass of \gls{SNF} for disposal, 
and minimizing both the \gls{HALEU} \gls{SWU} and the \gls{SNF} 
mass. 

\section{Once-through optimization}

\section{Recycling optimization}

Optimization for this work means minimizing certain material requirements, 
specifically the \gls{HALEU} mass and the \hl{???}. This work performs three 
different optimization problems: minimize the \gls{HALEU} mass, minimize the 
\hl{???}, and minimizing both. 

\cite{andrianov_optimization_2019} minimized total discounted costs using 
the MESSAGE energy planning software. 

I get a general sense that a lot of work has been done to optimize fuel cycles 
for cost, unf management, or proliferation risk. Not seeing a whole lot on 
material requirements. 
Methods I'm seeing: genetic algorithms, linear programming, stochastic linear 
programming (same thing?)
