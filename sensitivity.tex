\section{Sensitivity Analysis}
Sensitivity analysis identifies how the variation of model input
parameters affect select performance 
metrics of the fuel cycle. Additionally, it reveals the combined effect 
of varying multiple parameters on 
an output metric. This information is then used to design 
an optimized transition scenario by identifying which input parameters 
will affect the results the most, and how these parameters should be 
changed to obtain desired results (e.g., minimizing \gls{HALEU} requirements
of a transition). 

\subsection{Methodology}
The analysis was performed on Scenario 7 and Scenario 
14 to provide insight into how the input parameters affect the output 
metrics for both a once-through and recycle fuel cycle. 

\Cyclus was coupled to Dakota \cite{adams_dakota_2019},
and Dakota is the driver for the analysis. This coupling mirrors  
the scripts in the \texttt{dcwrapper} GitHub repository \cite{chee_arfc/dcwrapper_2019}.
Three different types 
of sensitivity analysis were performed: \acrfull{OAT}, synergistic, 
and global. \gls{OAT} analysis varies a single input parameter to 
investigate the effect of each parameter individually. In the context of 
this work, synergistic 
analysis varies two input parameters at once to investigate how the 
interaction of the two parameters affects the results. Finally, the global 
analysis varies more than two input parameters to provide a holistic 
view of how multiple parameters interact and affect the output metrics. 

The input parameters varied include the transition 
start time, the build share of each type of advanced reactor, 
the \gls{LWR} lifetime, and the discharge burnup of fuel from the 
\gls{HALEU}-fueled reactors. The transition start time ranges from January 
2025 to January 2040 in three-month intervals, but the same energy demand 
will be specified for all perturbations (87.20 GWe-yr). The build share 
variable will be repeated three times (for varying the build share of each 
type of advanced reactor) with build share percentages ranging from 0-50\% 
in increments of 5\%. To account for the build share of an advanced reactor
the deployment scheme described in Section \ref{sec:once-through-methods}
was adjusted. Instead of the reactor type with the largest power output 
deployed first, the reactor with the specified build share is deployed first 
until the build share is met. Then the remaining two reactor types are 
deployed in the manner described in Section \ref{sec:once-through-methods},
with the larger of the two reactors preferentially deployed and the smaller 
reactor deployed last to meet the power demand. Figure \ref{fig:build-share-deploy}
illustrates how this deployment scheme is applied to meet a demand of 530 
MWe and a VOYGR build share of 50\%. 

\begin{figure}
    \centering 
    \includegraphics[scale=0.45, trim=50 100 100 0,clip]{VOYGR_build_share.pdf}
    \caption{Demonstration of the adjusted advanced reactor deployment 
    scheme to meet a demand of 530 MWe and a VOYGR build share of 
    50\%.}
    \label{fig:build-share-deploy}
\end{figure}

The \gls{LWR} lifetimes are varied based 
on the percent of the fleet that operate for 60 and 80 years. This 
variable will vary between 0-50\% of the \gls{LWR} fleet operating for 80 
years, while the other \glspl{LWR} operate for 60 years. The 
\glspl{LWR} do not all start operation at the same time, so the 
selection of the \glspl{LWR} that operate for 80 years affects the results, 
even if the number is the same. Therefore, reactors will be selected for 
operation to 80 years in 
descending order of power output, reflecting the greater likelihood of 
larger units receiving a license extension. Previous sensitivity analysis of 
fuel cycle transitions considered the impact of the transition start time 
and the \gls{LWR} lifetimes \cite{chee_sensitivity_2019,feng_sensitivity_2020},
which forms the basis for why these input parameters were selected for this 
work.

Variations of the discharge burnup of each \gls{HALEU}-fueled reactor (the 
Xe-100 and \gls{MMR}) were carried out slightly differently. Variation 
of this parameter is being done because these reactors are aiming to achieve 
burnups that are much higher than what the NRC has approved fuel for 
\hl{Citation?}. Therefore, inclusion of this variable explores the impact 
on resource needs for if these reactors are not able to achieve their 
reported burnups. This analysis is not performed for the VOYGR because this 
design aims to achieve a burnup that the NRC is comfortable with. 

To vary the burnup of the Xe-100, two different systems were enacted. The 
first was to vary the number of passes each pebbles goes through the core
while keeping the length of each pass and the total mass of uranium 
in the core constant. Varying the number of passes between two and six 
results in discharge burnups of 56, 84, 112, 140, and 168 MWd/kgU, 
using Eq. \ref{eq:fuel_mass}. The second system was to vary the length 
of each pass while keeping the number of passes and the total mass of uranium 
constant. The goal of the second system is to cause a small change in the 
discharge burnup to examine how a smaller change in the burnup affects 
the resource needs. This system results in burnups of 151 and 185 MWd/kgU, 
which are about $\pm$10\% of the stated discharge burnup of 168 MWd/kgU. 

To vary the burnup of the \gls{MMR} the cycle time, and thus lifetime, of 
the reactor was varied while the total mass of uranium was held constant. 
Cycle times of 10, 15, and 20 years were considered, resulting in burnup 
values of 41, 62, and 82 MWd/kgU. Additionally, the lifetime was varied to 
result in burnups $\pm$10\% and $\pm$5\% around the stated burnup of 
82 MWd/kg, resulting in burnups of 74, 78, 86, and 90 MWd/kgU. 

The output metrics of 
interest for this analysis include the amount of waste generated that 
must be sent to a repository (\gls{SNF} mass), the mass of enriched uranium, 
the mass of \gls{HALEU},
the amount of \gls{SWU} capacity required to produce all enriched uranium, the 
\gls{SWU} capacity required to produce \gls{HALEU}, and the feed uranium 
required to produce \gls{HALEU}. Each metric will be evaluated based on the 
cumulative sum required, starting at the transition start time. Previous 
sensitivity analysis of fuel cycle transitions considered the waste 
discharged, \gls{SWU} 
capacity required, and natural uranium requirements
\cite{richards_application_2021,feng_sensitivity_2020} 


\subsection{One-at-a-time}
\subsubsection{Scenario 7}
This section discusses the results of the \gls{OAT} sensitivity analysis 
as applied to Scenario 7. Many of these results are discussed in 
\hl{cite ANS paper}. 

Delays in the transition start time generally decrease all of the output 
metrics, as shown in Figure \ref{fig:ts_scenario7}. This results matches 
expectations, because if the advanced reactors are deployed for less time, 
then they require fewer resources. However, there are some increases in 
the output metrics between individual time steps. For example, a transition 
start time of October 2029 increases the fuel mass and \gls{SNF} mass 
relative to July 2029. This increase is the result of changes in the number 
of each advanced reactor deployed, because of differences in the gap 
between energy produced and energy demand that must initially be filled when 
advanced reactors are deployed. By waiting until October 2029, more VOYGRs 
are deployed than when the transition starts in July 2029. As discussed in 
Chapter \ref{ch:once_through_results}, the VOYGR needs a larger fuel mass, 
and thus discharges more \gls{SNF} than the Xe-100 and \gls{MMR}. Therefore, 
these two metrics increase for this particular transition start time. 

\begin{figure}
    \centering
    \includegraphics{ts.pdf}
    \caption{Change in each metric as a function of transition start 
    time, relative to a transition start in January 2025.}
    \label{fig:ts_scenario7}
\end{figure}

\begin{figure}
    \centering
    \includegraphics{lwr.pdf}
    \caption{Change in each metric as a function of percent of LWR fleet  
    operating for 80 years, relative to 0\%.}
    \label{fig:lwr_scenario7}
\end{figure}
\begin{figure}
    \centering
    \includegraphics{xe100.pdf}
    \caption{Change in each metric as a function of Xe-100 build share, 
    relative to a build share of 0\%.}
    \label{fig:xe100_scenario7}
\end{figure}

As the \gls{MMR} build share increases all of the metrics increase, as shown 
in Figure \ref{fig:mmr_scenario7}. One would expect some of the metrics 
to increase with the \gls{MMR} build share, such as the \gls{HALEU} metrics 
because the \gls{MMR} has a smaller specific power than the Xe-100 or the 
\gls{SWU} metrics because the \gls{MMR} requires the highest enrichment 
assay of the three advanced reactors. However, it is not intuitive that 
all of the metrics would increase. 
\begin{figure}
    \centering
    \includegraphics{mmr.pdf}
    \caption{Change in each metric as a function of MMR build share, 
    relative to a build share of 0\%.}
    \label{fig:mmr_scenario7}
\end{figure}

Examination of the number of each reactor type deployed as a function 
of time and build share (Figure \ref{fig:mmr_reactors_s7}) identifies 
that the deployment scheme of this work contributes to the increase 
in all of the metrics. As the \gls{MMR} build share increases, the number of 
\glspl{MMR} increases and the number of Xe-100s decreases, as one would 
expect. However, the number of VOYGRs increases with \gls{MMR} build share 
and the number of \glspl{MMR} peaks in each simulation. The number of 
\glspl{MMR} peaks around 2055, when the last of the \glspl{LWR} is decommissioned.
Prior to 2055, all of the advanced reactors deployed replace energy generated 
by the \glspl{LWR}. After 2055, any advanced reactors deployed replace 
energy generated from other advanced reactors. The \gls{MMR} has a 20 year 
lifetime, while the Xe-100 and VOYGR have a 60 year lifetime. Therefore during 
the years simulated, \glspl{MMR} will be decommissioned but Xe-100s and VOYGRs 
aren't. The power replaced after 2055 is not replaced on a 1:1 rate by 
reactor type, at most only 50\% of the energy previously generated by 
\glspl{MMR} is replaced by new \glspl{MMR}, and the rest is replaced by VOYGRs 
based on the deployment scheme. This leads to the peak in the number of 
\glspl{MMR} and the increase in the number of VOYGRs as a function of 
\gls{MMR} build share. 

\begin{figure}
    \centering
    \includegraphics{mmr_combined_reactors.pdf}
    \caption{Number of Xe-100s (top left), MMRs (top right), and 
    VOYGRs (bottom left) deployed as a function of time and 
    MMR build share.}
    \label{fig:mmr_reactors_s7}
\end{figure}

When the VOYGR build share was varied, the impact on the metrics was 
opposite that of when varying the Xe-100 build share (Figure 
\ref{fig:voygr_scenario7}). The total fuel mass and \gls{SNF} mass both 
increase, the \gls{SWU} capacity is relatively constant, and the 
\gls{HALEU} metrics all decrease. The change in each of these metrics 
occurs because as the VOYGR build share increases, the number of Xe-100s 
decreases and the number of \glspl{MMR} is relatively constant. Therefore 
there is a replacement of Xe-100s with VOYGRs with increasing build share, 
the opposite of what happens with an increasing Xe-100 build share. 

\begin{figure}
    \centering
    \includegraphics{voygr.pdf}
    \caption{Change in each metric as a function of VOYGR build share, 
    relative to a build share of 0\%.}
    \label{fig:voygr_scenario7}
\end{figure}

\subsubsection{Scenario 14}

\subsection{Synergistic}
\subsubsection{Scenario 7}

\subsubsection{Scenario 14}

\subsection{Global}
\subsubsection{Scenario 7}

\subsubsection{Scenario 14}

\section{Optimization}