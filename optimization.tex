\section{Methodology}
Optimization of the nuclear fuel cycle is intended to develop 
a fuel cycle based on a specific objective or objectives. For this 
work, we developed three different optimization problems to apply to 
Scenarios 7 (no growth, once-through transition to the \gls{MMR}, Xe-100, 
and VOYGR) and 14 (no growth, limited recycling transition to the 
\gls{MMR}, Xe-100, and VOYGR): minimizing the \gls{SWU} capacity to 
produce \gls{HALEU}, minimizing the mass of \gls{SNF} for disposal, 
and minimizing both the \gls{HALEU} \gls{SWU} and the \gls{SNF} 
mass. To optimize these transitions, we consider six different 
variables: percent of \glspl{LWR} operating for 80 years, the build share 
of Xe-100s, the build share of \glspl{MMR}, the build share of VOYGRs, 
the discharge burnup of the Xe-100, and the discharge burnup of the 
\gls{MMR}. All of these variables were considered in the sensitivity 
analysis. The transition start time was also considered in the sensitivity 
analysis. However, that analysis showed that the transition start time has
a much smaller impact on each of the metrics and delaying the transition 
can lead to unfulfilled energy demand. Therefore, this parameter is 
not considered in the optimization of each scenario. 

For this work, the percent of \glspl{LWR} operating for 80 years 
was constrained to between 0-50\% of the current \gls{LWR} fleet. 
The build share of each advanced reactor was allowed to range between 
0-100\%, but the three parameters had to sum to 100\%. The discharge burnups 
are restricted to the values considered in the sensitivity analysis.

We coupled \Cyclus \cite{huff_fundamental_2016} with Dakota 
\cite{adams_dakota_2019} to perform this optimization. For the 
single-objective problems, we used the ``soga'' solve method in 
Dakota, which is a single-objective genetic algorithm. For the 
multi-objective problems we used the ``moga'' solve method in 
Dakota, which is a multi-objective genetic algorithm. Genetic algorithms
have previously been used to optimize fuel cycle transitions 
\cite{passerini_systematic_2014}, which led to the decision to use 
them for this work. Each of these methods have multiple 
hyperparameters that require careful tuning and selection. 

\subsection{Hyperparameter tuning}
Genetic algorithms have a variety of hyperparameters, such as the the 
population size, mutation rate, and crossover rate, that can affect 
the results of the algorithm. Therefore, we tuned multiple hyperparameters
to determine the best combination for this work and the exact algorithm 
used by Dakota. We performed the tuning by performing a grid search across 
the possible values for different parameters, a method recommended 
by Deb \cite{deb_multi-objective_2001}. Not all hyperparameters or 
all possible values of the hyperparameters were considered. We downselected from 
the possible hyperparameters and values defined in the Dakota Reference 
Manual based on personal intuition and limitations on time to perform 
the tuning. The 
first grid search performed 40 iterations of different hyperparameter 
values across a coarse grid. The total number of samples was restricted to 
500 for each evaluation, which restricts the population size and the 
number of generations, as the product of these two must equal the 
total number of samples.Table \ref{tab:soga_tuning} describes 
the hyperparameters considered in the tuning and the possible 
values they take. Some of the variables, such as the mutation 
type, take discrete values while others, such as the crossover 
rate, can take continuous values. The crossover rate, mutation 
rate, and constraint penalty were treated as continuous for this 
tuning, and therefore were randomly sampled within the defined range. 
The population size can be a continuous variable, but was 
treated as discrete for this initial search to provide a well-defined 
number of generations.

\begin{table}
    \centering 
    \caption{Hyperparameters and values considered in tuning. If a range 
    of values is provided, then the hyperparameter a random value in 
    that range was selected.}
    \label{tab:soga_tuning}
    \begin{tabular}{c c c}
        \hline
        Hyperparameter & Coarse Search & Fine Search \\
        \hline 
        Experiments [\#] & 1-40 & ?? \\
        Population size [\#] & 5, 10, 25, 50, 100 & \\
        Constraint penalty & 0.5 $\leq$ x $\leq$ 2 & \\
        Crossover rate & 0.1 $\leq$ x $\leq$ 0.9 & \\
        Mutation type & replace uniform, offset normal & \\
        Mutation rate & 0.01 $\leq$ x $\leq$ 0.2 & \\
        \hline       
    \end{tabular}
\end{table}




Based on the tuning performed, we selected values for each of the 
hyperparameters. Table \ref{tab:ga_parameters} defines the 
parameters used with the optimization schemes.

\begin{table}
    \centering
    \caption{Parameters of genetic algorithm models}
    \label{tab:ga_parameters}
    \begin{tabular}{c c}
        \hline
        Parameter & Value \\
        \hline
        Population & 20 \\
        Constraint penalty & \\
        Crossover rate & 0.8 \\
        Mutation type & normal distribution \\
        Mutation rate & 0.2 \\
        \hline
        
    \end{tabular}
\end{table}

\section{Once-through optimization}
\subsection{Minimize HALEU SWU}

\subsection{Minimize waste mass}

\subsection{Minimize HALEU SWU and waste mass}

\section{Recycling optimization}
\subsection{Minimize HALEU SWU}

\subsection{Minimize waste mass}

\subsection{Minimize HALEU SWU and waste mass}



\cite{andrianov_optimization_2019} minimized total discounted costs using 
the MESSAGE energy planning software. 

I get a general sense that a lot of work has been done to optimize fuel cycles 
for cost, unf management, or proliferation risk. Not seeing a whole lot on 
material requirements. 
Methods I'm seeing: genetic algorithms, linear programming, stochastic linear 
programming (same thing?)
