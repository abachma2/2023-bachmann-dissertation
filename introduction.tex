\section{Motivation}
Nuclear energy supplied about 19\% of all energy and over 50\%  
of all carbon-free energy in the US in 2021 
\cite{energy_information_administration_electricity_2022}, making nuclear 
energy the third largest producer of energy and the largest 
producer of carbon-free energy in the United States. Because of its role 
as a carbon free supply of base-load power, nuclear energy is expected 
to play a role in meeting carbon emission goals. Two reactor designs 
comprise the current nuclear reactor fleet: \glspl{PWR} and \glspl{BWR}, 
which both fall under the classification of \glspl{LWR}.
Both designs use uranium dioxide pellets as fuel, with the uranium 
enriched to no more than 5\% $^{235}$U. These types of reactors 
have commercially operated in the US since 1957. The \glspl{LWR} 
currently deployed will be retiring over the next 30 years, with the 
last license expiring in 2055 \cite{nei_us_2021}. Therefore, if 
nuclear energy is to continue to produce energy in the US 
and assist in meeting carbon emission goals, we must deploy new reactors. 

Multiple countries around the world are building new \glspl{LWR}
\cite{world_nuclear_association_plans_2022}, and developing
new reactor designs, often called advanced reactors 
\cite{hussain_advances_2018}, to 
replace or expand the current fleet of reactors. These advanced reactors 
cover a large swath of design space, with wider ranges in: energy output, 
fuel form, and cycle length. The 
\gls{DOE} \gls{ARDP} \cite{noauthor_advanced_nodate}
supports research and building demonstration reactors for a select set of 
the new designs to advance the deployment studies in the US. These 
demonstration reactors are planned to be complete by the late 2020's.

One important question that has arisen from the \gls{ARDP} is how the 
development of supply chains can support the nuclear fuel cycle of 
advanced reactors. 
One key difference in the fuel for these advanced 
reactor designs is the enrichment 
level of the fuel. Many advanced reactor designs require  
\acrfull{HALEU}, which is uranium enriched between 5\% and 20\% $^{235}$U.
There is no commercial supply of 
uranium enriched to this level in the US, therefore the \gls{DOE} 
is investigating how to develop a nuclear fuel cycle for uranium enriched to this 
increased level. 

Deploying advanced reactors that require \gls{HALEU} is expected to have 
large impacts on the \gls{NFC} in the US; specifically in the development 
of infrastructure and supply chains to support these reactors. To assist 
in understanding the impacts 
from deploying \gls{HALEU}-fueled reactors, modeling the \gls{NFC} 
can provide the material requirements of potential transition scenarios. 
The \gls{NFC} describes how fuel for fission energy systems is 
prepared, processed, and disposed \cite{tsoulfanidis_nuclear_2013}. 
The technology employed in the \gls{NFC} often defines the fuel 
cycle type, 
and a transition in the nuclear fuel cycle occurs when 
introducing new fuel types or other fuel cycle technologies.  
Modeling  a \gls{NFC}, typically aided by a fuel cycle simulator, 
includes the deployment and decommissioning of facilities in the \gls{NFC}, 
such as mines or reactors, and the materials traded between facilities. 
\gls{NFC} simulators have been used to model a variety of \gls{NFC} 
transitions \cite{sunny_transition_2015,bae_fuel_2018,piet_dynamic_2011} 
and quantify the resource requirements of \glspl{NFC} transitions
\cite{bachmann_enrichment_2021}. 

There are two methods to produce \gls{HALEU}: enrich natural uranium 
or downblend \acrfull{HEU} to achieve the desired enrichment level. The current US
nuclear commercial fuel cycle relies on enrichment of natural uranium 
to create fuel for \glspl{LWR}, but does not have the capability to enrich
natural uranium to produce \gls{HALEU} \cite{nuclear_energy_institute_addressing_2018}.  
The amount of available natural uranium and facility capacities, such 
as the facility throughputs and \gls{SWU} capacity, limit 
\gls{HALEU} production through enrichment. 
The \gls{HEU} being considered for downblending comes from three different 
\gls{DOE} sources: the spent \gls{EBR} fuel stockpiles at \gls{INL} 
\cite{patterson_haleu_2019}, stockpiles at \gls{SRS} \cite{regalbuto_addressing_2020}, 
and stockpiles at Y-12 National Security Complex 
\cite{robinson_establishment_2020}. The size of each stockpile limits the amount 
of \gls{HALEU} produced through downblending, and each stockpile is capable 
of producing no more than 20 MTU of \gls{HALEU}. Additionally, two 
of the potential stockpiles of \gls{HEU} to downblend  
contain impurities that would affect reactor performance 
\cite{vaden_isotopic_2018,nelson_foreign_2010}.
These impurities may limit the amount of downblended \gls{HEU} that a reactor 
a reactor can use at once. 
Currently, there is only one facility in the US commercially licensed to 
downblend \gls{HEU}, the BWXT Nuclear Fuel Services Inc. facility in 
Erwin, TN, which is expected to have the capacity to downblend 1-2 
MT of \gls{HEU}, producing up to 10 MT of \gls{HALEU} each year \cite{nagley_ha-leu_2020}.

One limitation in modeling the \gls{NFC} is the large array of potential 
input parameters, such as when to start a transition or the speed of the 
transition. To understand 
the impact of input parameters on the output metrics (e.g., material 
requirements) of a fuel cycle, fuel cycle modelers perform sensitivity 
analysis on the 
modeled \gls{NFC}. Sensitivity analysis involves 
modeling a fuel cycle with small perturbations in various input 
parameters and analyzing the variance or spread of select output metrics. 
Sensitivity analysis has been used in multiple \gls{NFC} analysis 
\cite{chee_sensitivity_2019,feng_sensitivity_2020,thiolliere_methodology_2018}
to identify the model parameter or parameters that have the greatest 
impact on specific model outputs or material requirements. Sensitivity 
analysis can also reveal underlying information about a system that is not 
necessarily intuitive, such as how much the modeling methodology 
affects the results. Understanding the 
material requirements of a \gls{NFC} coupled with sensitivity analysis 
aids in optimizing the \gls{NFC} based on given criteria, such as 
minimizing the amount of \gls{HALEU} required. Various 
optimization schemes have been applied to \glspl{NFC} to optimize the fuel 
cycle based on criteria such as fuel requirements \cite{kim_selection_1999},
waste production \cite{shwageraus_optimization_2003}, and combinations 
of these metrics in multi-objective 
problems \cite{passerini_systematic_2014}.

\section{Research Goals}
The goal of this work is to investigate the impacts of deploying reactors 
fueled 
by \gls{HALEU} in the United States, including the impacts that the reactors 
have on the \gls{NFC} and the impacts the \gls{NFC} has on the reactors. 
The results of this work are intended to
aid and guide policy makers and key stake holders on how to best establish a 
fuel cycle to support the deployment of \gls{HALEU}-fueled reactors. 
Within this primary goal, there are three specific objectives:
\vspace{0.2cm} 
\noindent
\begin{enumerate}
\item Quantify potential material requirements for the transition 
from \glspl{LWR} to advanced reactors in open and closed 
fuel cycles.

\item Understand the impacts of fuel cycle parameters on the material 
requirements and design optimized transition scenarios.

\item Identify potential limitations in using downblended \gls{HEU} 
to produce \gls{HALEU} based on reactor performance.

\end{enumerate}
The goals of this work will be completed through the following steps:
\vspace{0.2cm} 
\noindent
\begin{enumerate}
\item \textbf{Model the transition from \glspl{LWR} to advanced reactors.} 
Transition scenarios to multiple fuel cycles with \gls{HALEU}-fueled 
reactors will 
be modeled using the \gls{NFC} simulator \Cyclus \cite{huff_fundamental_2016}. 
Multiple transitions will be modeled, with each one varying based on the type
of  
advanced reactors, energy demand, and fuel cycle option (open or closed). This 
step will quantify and compare the material requirements of each transition 
scenario to understand how the material requirements compare to what is available 
through the current US fuel cycle and published estimates of potential 
\gls{HALEU} needs. 

\item \textbf{Perform sensitivity analysis and optimize select transition scenarios.}
Sensitivity analysis will be performed on a subset of the transition scenarios
modeled by coupling \Cyclus with Dakota \cite{adams_dakota_2019}. This step 
will identify the most impactful model parameters on 
the material requirements of the fuel cycle. These results will then be 
used to determine optimized transition scenarios to minimize specific 
material requirements. 

\item \textbf{Neutronics analysis of \gls{HALEU} created from impure \gls{HEU}.}
The neutronics of the \gls{HALEU}-fueled reactors in the transition scenarios 
will be modeled to identify potential effects of using \gls{HALEU} produced 
from downblended \gls{HEU} with known impurities. This step will investigate 
potential limitations in using this \gls{HALEU} production method and 
how different \gls{HALEU} production methods affect reactor operation and 
key safety parameters, such as neutron multiplicity, delayed neutron 
fractions, and reactivity coefficients.
\end{enumerate}


The following chapters present preliminary and future work to complete 
each of these steps and complete the objectives of this work. Chapter 2 
provides some 
background information and literature review on the nuclear fuel cycle,
nuclear fuel cycle 
modeling efforts, and the use of \gls{HALEU} in reactors.
Chapter 3 discusses the methodology for modeling the transition to advanced 
reactors in a once-through fuel cycle. This chapter includes the scenarios
modeled, information on the reactors included in the transitions, and the 
flow of 
material in each scenario. Chapter 4 discusses the results of the 
once-through fuel cycle transition scenarios. Chapter 5 summarizes key 
takeaways from the previous chapters and the proposed future work for 
this project. 