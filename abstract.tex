The US is looking into the deployment of advanced reactors that 
require uranium enriched between 5-20\% $^{235}$U, often referred to 
as \gls{HALEU}. There are no commercial facilities 
in the US to produce \gls{HALEU}, 
prompting questions of how to create a dependable supply chain of \gls{HALEU}. 
\gls{HALEU} can be produced through two primary methods: downblending 
\gls{HEU} and enriching natural uranium. The amount of \gls{HEU} available 
and impurities present in the \gls{HEU} limit downblending capabilties.
The \gls{SWU} capacity and amount of natural uranium available limit 
enriching natural uranium capabilities. To understand the resources necessary 
to commercially produce \gls{HALEU} with each of these methods, we can 
quantify the material requirements of transitioning to \gls{HALEU}-fueled 
reactors. 

We model the transition from \glspl{LWR} to different 
advanced reactors, considering once-through and closed 
fuel cycles. We use \Cyclus and open-source 
information about \glspl{LWR}, the X-energy Xe-100, the
\gls{USNC} \gls{MMR}, and the NuScale VOYGR to model these 
transitions. To more accurately model the closed fuel 
cycles, we develop a new \Cyclus archetype, called OpenMCyclus, 
that couples with OpenMC to dynamically model fuel depletion in a 
reactor.  
The results of this transition analysis show how the 
characteristics of the advanced reactors deployed drive 
the materials required to support the fuel cycle. Closing the 
fuel cycle reduces the materials required, but the 
reduction in materials is driven by the amount of material 
available for reprocessing. 

The transition analysis also show that many of the transition 
parameters affect  the material requirements 
of the transition. Therefore, we perform sensitivity 
analysis on one of the once-through transitions by coupling 
\Cyclus with Dakota. The results of 
the sensitivity analysis highlight some of the trade-offs between 
different reactor designs. One such tradeoff is the increased 
\gls{HALEU} demand but decreased spent fuel discharged when increasing 
the Xe-100 deployment and decreasing the VOYGR  deployment. Additionally, 
these results identify the Xe-100 discharge burnup as consistently 
being one of the most impactful input parameters for this transition, 
because of how the deployment scheme affects the number of 
Xe-100s built no matter which advanced reactor build share is 
specified. 

We then use the \Cyclus-Dakota to optimize a once-through transition, 
using the genetic algorithms in Dakota. In single-objective problems to 
minimize the \gls{SWU} capacity required to produce \gls{HALEU} and 
minimize the amount of spent nuclear fuel, the algorithm finds 
solutions that are consistent with the results of the 
sensitivity analysis. However, the results cannot be taken at face 
value, because the algorithm did not fully converge and genetic 
algorithms do not strictly enforce the 
linear constraint for the advanced reactor build shares to sum to 
100\%. Similar results occur when using 
this method for a multi-objective problem to minimize both objectives 
previously considered. 

Finally, we use models of the Xe-100 and \gls{MMR} reactors to 
evaluate the performance of downblended \gls{HEU} in these 
two reactor designs. We compare the performance of the downblended 
\gls{HEU} to nominally enriched fuel, based on the 
\keff, \betaEff, energy- and spatially-dependent neutron 
fluxes, as well as the fuel, moderator, coolant, and total reactivity 
temperature feedback coefficients. The differences in the fuel 
compositions leads to differences in each of the metrics. 
However, these differences are within error or would not prevent the 
reactor from meeting stated design specifications or operating 
in a safe state. 

The work completed in this dissertation demonstrates a methodology for 
understanding the effects of deploying \gls{HALEU}-fueled reactors 
in the US. The effects investigated here include the materials and 
resources required to support these reactors, and how the 
parameters of the transition affect these requirements. This 
information can be used to develop the necessary infrastructure 
and supply chains. Futhermore, this work explores how the 
\gls{HALEU} production method (enriching compared with downblending)
affects reactor performance. 
