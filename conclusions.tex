% General Summary

%Once-through analysis

% Closed fuel cycle analysis

% Sensitivity analysis
The analysis in this chapter examined the effects of different input parameters for 
their effects on different output metrics for the fuel cycle transition from 
\glspl{LWR} to different advanced reactors. The \gls{OAT} analysis identified the 
trends from varying each input parameter independently, and how the deployment 
scheme for this work impacts each of the results. The analysis also 
identified tradeoffs between different reactor designs, such as how the Xe-100 
needs more \gls{HALEU} than the VOYGR, but a smaller fuel mass. The synergistic 
analysis identified how some of the input parameters interact to affect the output 
metrics. While some of the combinations of input parameters (e.g., varying 
the \gls{LWR} lifetime and the VOYGR build share) had consistent trends 
across the input parameter space, some combinations (e.g., varying the Xe-100 
burnup and the \gls{MMR} share) varied in their effect across the input parameter 
space. Finally, the global sensitivity analysis quantified the effect of different 
input parameters on the output metrics. We build surrogate models to fully capture 
the input parameter space. The models were consistent with the Sobol' indices, 
such that the Xe-100 burnup is the most impactful parameter when varying any 
of the three advanced reactor build shares. 

% Optimization

% Neutronics

\section{Future Work}