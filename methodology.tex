Modeling the transition between different nuclear fuel cycles provides 
information on the quantity and timing of different materials to meet 
specific objectives, such as providing fuel for reactors to meet a 
prescribed energy demand. 
The fuel cycle scenarios created show the resources required for the transition from the 
current fleet of \glspl{LWR} in the US to advanced reactors, with a focus on 
advanced reactors requiring \gls{HALEU}.
\Cyclus \cite{huff_fundamental_2016} simulated all of the transitions in this 
work, with each of the fuel cycle facilities (including reactors) defined through 
the \Cycamore archetype library \cite{carlsen_cycamore_2014}. Each simulation models 
current 
\glspl{LWR} starting in 1965 and models all reactors out to 2090 with a timestep 
of one month. The \gls{IAEA} \gls{PRIS} database \cite{noauthor_power_1989} 
provided the start and select end dates for \glspl{LWR}. The \gls{PRIS} database 
only contains end dates for reactors shut down before the publication of the 
database each year. Reactors still in 
operation in December 2020, and thus 
lacking an end date in the \gls{PRIS} database, are assumed to operate until their 
current operating license expiration date
\cite{noauthor_us_nodate}.
Only reactors in the \gls{PRIS} database with a power level above 400 MWe 
were used in the simulation 
to avoid including prototype and research reactors present in the database. 
Approximate masses for fuel used in the cores of the \glspl{LWR} were obtained 
from \cite{todreas_nuclear_2012} and \cite{cacuci_handbook_2010}. 

Recipes define the composition of each material in the simulation. 
Recipes for \gls{LWR} fresh and spent fuel were found in \cite{jacobson_verifiable_2010}.
Section \ref{sec:reactor_methods} describes we obtained recipes for fresh 
and spent fuel of the advanced reactors. The tails assay from enrichment 
is defined as 0.2\% assay (weight fraction $^{235}$U). 

This work models the transition to advanced reactors assuming a 
once-through fuel cycle. The models are defined based 
on the energy demand 
of the scenario (either no growth or 1\% annual growth in demand) and by the 
advanced reactors deployed in the scenario. The transition from \glspl{LWR} to 
advanced reactors begins in January 2025, so the energy supplied by 
\glspl{LWR} in 2025 is the basis for the energy demand in each scenario.  
Many of the 
current plans for \gls{HALEU}-fueled reactors do not have these reactors deployed 
until the late 2020s \cite{nichol_current_2021} through \gls{ARDP} and 
other programs. We selected 2025 as the transition 
start time for this 
work because it will provide an upper bounding case for if the US is to deploy 
these reactors in mass on a more aggressive timeline. 

This work compares the fuel cycle scenarios on various metrics: 
how well the energy demand is met,
the number of reactors deployed, the 
uranium usage (both feed uranium to produce enriched uranium and enriched 
uranium sent to the reactors), 
the \gls{SWU} capacity required to produce the enriched uranium, and the 
amount of waste produced. The natural 
uranium usage and waste production are two of the metrics used in the \gls{ES} 
\cite{wigeland_nuclear_2014}, and the mass of enriched uranium was the 
primary result of \cite{dixon_estimated_2022}.

\section{Reactors} \label{sec:reactor_methods}
This work considers three advanced reactors: the \gls{USNC} \gls{MMR}, 
\cite{noauthor_usnc_2021}, the X-energy Xe-100 
\cite{mulder_overview_2021}, and 
the NuScale VOYGR reactor
\cite{nuscale_chapter_2020-1,reyes_nuscale_2021,reyes_correction_2022}. The 
\gls{USNC} \gls{MMR}
and X-energy Xe-100 reactors require \gls{HALEU}, but the NuScale VOYGR requires
a similar enrichment level to current \gls{LWR} fuel (Table \ref{tab:reactor_summary}). 
This work incldues the NuScale VOGYR, despite not requiring \gls{HALEU} fuel, 
because the \gls{NRC} granted it design approval \cite{world_nuclear_news_nuscale_2021} 
and it is very likely to be 
deployed along-side reactors that require \gls{HALEU}. Including the NuScale 
VOYGR reactor in the transition scenarios provides insight into how the deployment 
of \gls{HALEU}-fueled and non\gls{HALEU}-fueled advanced reactors in tandem 
affects the material requirements of the transition. However, because the 
NuScale VOYGR reactor does not require 
\gls{HALEU}, this work does not consider the transition from \glspl{LWR} 
to only the VOYGR.

\begin{table}[ht]
    \centering
    \caption{Advanced reactor design specifications.}
    \label{tab:reactor_summary}
    \renewcommand{\arraystretch}{1.5}
    \begin{tabular}{p{3.5cm}p{3cm}p{3cm}p{3cm}}
        \hline
        Design Criteria & \gls{USNC} \gls{MMR} \cite{noauthor_usnc_2021} & 
        X-energy Xe-100 \cite{mulder_overview_2021} & NuScale VOYGR 
        \cite{nuscale_chapter_2020-1,reyes_nuscale_2021,reyes_correction_2022}\\
        \hline
        Reactor type & Modular HTGR & Modular HTGR & SMR\\
        Power Output (MWth) & 15 & 200  & 250 \\
        Enrichment (\% $^{235}U$) & 19.75 & 15.5 & <4.95 \\
        Cycle Length (yrs) & 20 & online refuel & 2\\
        Fuel form & UO$_2$ \gls{FCM} compacts & UCO \gls{TRISO} pebbles & UO$_2$ pellets\\
        Discharge fuel burnup (GWd/MTU) & 82 & 
        168  & 45 \\
        Reactor Lifetime (yrs)& 20 & 60 & 60 \\
        \hline
    \end{tabular}
\end{table}

Defining a reactor with the \Cycamore \texttt{Reactor} archetype 
\cite{scopatz_cyclus_2015}, 
requires knowing the refueling scheme and the fuel composition for 
the reactor. The refueling scheme includes 
the cycle time, the refueling time, and the mass of fuel put into the reactor 
at each refueling. The parameters for the reactors (Table \ref{tab:reactor_summary})
are meant to closely match the design information about each reactor that 
is available through open-source information. The fuel mass required by 
each reactor type was calaculated based on the reactor thermal power, 
cycle length, and burnup:

\begin{equation}
    \text{mass [kg] = }\frac{\text{Power [MWth]}* \text{cycle 
    length [d]}}{\text{burnup [MWd/kg]}}
    \label{eq:fuel_mass}
\end{equation}

\noindent Eq.\ref{eq:fuel_mass} calculates the mass of uranium required 
in the core. To calculate the total mass of the fuel (including the carbon 
and/or oxygen in the fuel) this value was divided by the mass fraction of 
uranium in the fuel form for the reactor. Any non-uranium components 
of the fuel, such as silicon-carbide in \gls{TRISO} particles, were 
not considered in the mass. This methodology assumes that the 
uranium and uranium-containing fuel components would be the limiting 
factor in the fuel cycle and other fuel components would be available as 
needed. 

The \gls{MMR} does not undergo refueling, the initial core
burns for the entire lifetime of the reactor \cite{mitchell_usnc_2020}. 
Therefore, the \gls{MMR}
is modeled to not include any refueling and has a cycle length that 
matches the reactor lifetime.  
The Xe-100 reactor undergoes online refueling operations, with each 
\gls{TRISO} pebble passing 
through the reactor six times before discharge \cite{mulder_overview_2021}. 
Every six months about 1/7th 
of the pebbles in the core are expected to be discharged. Online refueling 
cannot be explicitly modeled in \Cyclus, so the Xe-100 refueling is modeled 
as a replacement of 1/7th of the entire core every 6 months with no down time 
for refueling.  
The VOYGR reactor contains 37 fuel assemblies, with three different enrichment 
levels \cite{nuscale_chapter_2020-1}. Each refueling replaces 13 fuel assemblies, 
with the middle assembly replaced at every refueling, therefore this reactor 
is modeled as a replacement of 13/37th of the total core mass every 
refueling outage. 
The fuel enrichment used is an average of the assembly enrichments 
presented in \cite{nuscale_chapter_2020-1}. Refueling is assumed to 
take one month, because that is the refueling outage time 
modeled for the \glspl{LWR} and one month is the minimum time step in the 
simulations. Fresh fuel recipes for these reactors are based on the 
required fuel form for each reactor using the appropriate uranium isotope 
ratio for enrichment. 

\section{Once-through fuel cycle}
Figure \ref{fig:once-through_fuel_cycle} shows the flow of material through 
the modeled once-through fuel cycles. The once-through fuel cycle models 
material from the mine to final disposal in a repository (the 
``HLW Sink'' in Figure \ref{fig:once-through_fuel_cycle}). The ``advanced reactor'' 
node in Figure 
\ref{fig:once-through_fuel_cycle} represents any subset of the advanced reactors included 
in the scenario. This is a simplified version of the modeled fuel cycle in 
\cite{bachmann_enrichment_2021}, removing steps in the fuel cycle that 
do not affect the reported results. The institutions in the simulations 
govern the deployment of reactors. The \Cycamore \texttt{DeployInst} institution 
\cite{huff_fundamental_2016}
deploys and decommissions the \glspl{LWR} according to their start and end dates. 
The \Cycamore \texttt{ManagerInst} deploys the advanced reactors as needed to 
meet the energy demand of the scenario. The \Cycamore \texttt{GrowthRegion} 
defines the energy demand of the scenario. All of the 
agents in the simulation are in the same region, modeling that all facilities are
in the same country. No potential tariffs are modeled on the commodity 
transactions of the \gls{DRE}. 

\begin{figure}
    \centering
    \begin{tikzpicture}[node distance=1.5cm]
        \node (mine) [facility] {Uranium Mine};
        \node (enrichment) [facility, below of=mine]{Enrichment};
        \node (reactor) [facility, below of=enrichment]{Reactor};
        \node (adv_reactor) [transition, right of=reactor, xshift=3cm]{Advanced Reactor};
        \node (wetstorage) [facility, below of=reactor]{Cooling Pool};
        %\node (drystorage) [facility, below of=wetstorage]{Dry Storage};
        \node (cooling) [transition, below of=adv_reactor]{Cooling Pool};
        \node (sinkhlw) [facility, below of=wetstorage, xshift=2.5cm]{HLW Sink};
        \node (sinkllw) [facility, left of=enrichment, xshift=-3cm]{LLW Sink};

        \draw [arrow] (mine) -- node[anchor=east]{Natural U} (enrichment);  
        \draw [arrow] (enrichment) -- node[anchor=south]{Tails}(sinkllw);
        \draw [arrow] (enrichment) -- node[anchor=east]{Fresh UOX}(reactor);
        \draw [arrow] (enrichment) -| node[anchor=west]{Fresh HALEU and LEU}(adv_reactor);
        \draw [arrow] (reactor) -- node[anchor=east]{Spent UOX}(wetstorage);
        \draw [arrow] (wetstorage) |- node[anchor=east]{Cool Spent UOX}(sinkhlw);
        %\draw [arrow] (drystorage) |- node[anchor=east]{Casked Spent UOX}(sinkhlw);
        \draw [arrow] (adv_reactor) -- node[anchor=west]{Spent Fuel}(cooling);
        \draw [arrow] (cooling) |- node[anchor=west]{Cooled Spent HALEU and LEU}(sinkhlw);`'

        \end{tikzpicture}
    \caption{Fuel cycle facilities and material flow between facilities in the 
    once-through fuel cycles modeled. Facilities in 
    blue are used in all once-through scenarios, the facilities in red are added in
    at the transition start time in the transition scenarios.}
    \label{fig:once-through_fuel_cycle}
\end{figure}

The once-through scenarios model the current fleet of \glspl{LWR} in the 
US and the transition to multiple 
combinations of the advanced reactors and different energy demand scenarios, 
summarized in Table \ref{tab:scenarios_once-through}. Scenario 1 models
the \gls{LWR} fleet without the transition to any advanced reactor to provide 
a comparison with what has historically been needed to for a fuel cycle 
based on enrichments to less than 5\% $^{235}$U. A 1\% annual growth in 
demand (Scenarios 8-13) is less than what Dixon et al. \cite{dixon_estimated_2022} 
modeled
(1.2\%-2\%) but more than the average growth between 2020-2050 in the 
reference case of the 2022 \gls{EIA} Annual Energy Outlook 
\cite{energy_information_administration_annual_2022} (0.82\%), and 
provides middle-ground estimate on material requirements for a growing 
energy demand. 

\begin{table}[ht]
    \centering
    \caption{Summary of the once-through fuel cycle transition scenarios.}
    \label{tab:scenarios_once-through}
    \begin{tabular}{l l l}
            \hline
            Scenario number & Reactors present & Energy growth model\\\hline
            1 & \glspl{LWR} & N/A \\
            2 & \glspl{LWR} and \gls{MMR} & No growth \\
            3 & \glspl{LWR} and Xe-100 & No growth \\
            4 & \glspl{LWR}, Xe-100, and \gls{MMR}& No growth\\
            5 & \glspl{LWR}, \gls{MMR}, and VOYGR & No growth\\
            6 & \glspl{LWR}, Xe-100, and VOYGR & No growth\\
            7 & \glspl{LWR}, Xe-100, \gls{MMR}, and VOYGR & No growth\\
            8 & \glspl{LWR} and \gls{MMR}& 1\% growth \\
            9 & \glspl{LWR} and Xe-100 & 1\% growth\\
            10 & \glspl{LWR}, Xe-100, and \gls{MMR}& 1\% growth\\
            11 & \glspl{LWR}, \gls{MMR}, and VOYGR & 1\% growth\\
            12 & \glspl{LWR}, Xe-100, and VOYGR & 1\% growth\\
            13 & \glspl{LWR}, Xe-100, \gls{MMR}, and VOYGR & 1\% growth\\
            \hline
    \end{tabular}
\end{table}

\section{Calculation of results}
The results presented for the transition scenarios are from the
output of each \Cyclus simulation. Because the non-reactor facilities 
in the simulations (e.g., enrichment facility) do not have a specific cap on the 
amount of material they can hold, the material being traded by these facilities 
does not accurately capture the material requirements of the reactors. Therefore, 
the material requirements and waste of each scenario are calculated based on 
the material sent to and discharged from the reactors in each scenario. 

The mass of enriched uranium reported is the mass sent to the reactors 
at each time step. The \gls{SWU} required in each scenario is calculated based on the 
mass sent to the reactors, and does not reflect the capacity of an actual 
facility. The mass of feed uranium required to produce the enriched uranium 
is calculated based on the mass of product sent to the reactors and Eq. 
\ref{eq:enrichment}.