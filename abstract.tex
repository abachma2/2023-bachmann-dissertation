The US is looking into the deployment of advanced reactors that 
require uranium enriched between 5-20\% $^{235}$U, often referred to 
as \gls{HALEU}. There are no commercial facilities 
in the US to produce \gls{HALEU}, 
prompting questions of how to create a dependable supply chain of \gls{HALEU}. 
\gls{HALEU} can be produced through two primary methods: downblending 
\gls{HEU} and enriching natural uranium. The amount of \gls{HEU} available 
and impurities present in the \gls{HEU} limit downblending capabilties.
The \gls{SWU} capacity and amount of natural uranium available limit 
enriching natural uranium capabilities. To understand the resources necessary 
to commercially produce \gls{HALEU} with each of these methods, we can 
quantify the material requirements of transitioning to \gls{HALEU}-fueled 
reactors. 

The purpose of this work is to investigate the effects of deploying 
\gls{HALEU}-fueled reactors in the US, such as changes in the amount of 
natural uranium required in the fuel cycle. Preliminary effort 
models the transition from the
current fleet of \glspl{LWR} to different subsets of advanced reactors, 
the X-energy Xe-100, \gls{USNC} \gls{MMR}, and NuScale VOYGR,
in a once-through fuel cycle. The first two advanced reactors considered 
require \gls{HALEU} and 
the NuScale VOYGR is also included to understand 
how the deployment of a non-\gls{HALEU}-fueled reactor in tandem with 
\gls{HALEU}-fueled reactors affects the fuel cycle needs. The 
transitions modeled assume both a no growth and a 1\% 
annual growth in energy demand. Metrics of interest include the 
number of reactors built, the mass of enriched uranium, the mass 
of feed uranium to produce the enriched uranium, the \gls{SWU} 
capacity needed, and the amount of \gls{SNF} produced in each scenario. 

The results from analyzing the once-through transition scenarios show 
that the trends in each of the material requirements depend more on the 
reactors deployed in the scenario than the energy 
demand of the scenario. The number of reactors needed in each scenario 
scales with the power output of the reactors deployed in each scenario 
and the energy demand, 
with the transition to only the \gls{MMR} requiring the most reactors for 
each energy demand modeled. The scenarios that deploy only the \gls{MMR} 
also require the most resources to produce \gls{HALEU}
because of the large number of reactors needed, low specific power of the 
\gls{MMR}, and the high fuel enrichment. Deploying the VOYGR 
alongside the \gls{HALEU}-fueled reactors generally increases the 
mass of enriched uranium required and waste generated compared with the 
scenarios deploying 
only \gls{HALEU}-fueled reactors. The increase is a result of the 
VOYGR requiring more fuel in the core than the other two reactors. 
Deploying VOYGRs alondside the \gls{HALEU}-fueled reactors requires similar 
feed uranium and \gls{SWU} capacity to produce all enriched uranium compared 
to deploying only the \gls{HALEU}-fueled 
reactors. All \gls{HALEU} requirements
(mass of \gls{HALEU}, feed uranium, \gls{SWU} capacity, and \gls{HALEU} 
waste) decrease when deploying VOYGRs with \gls{HALEU}-fueled reactors 
because a portion of the energy generated is supplied by a 
non-\gls{HALEU}-fueled reactor. This scenarios provide valuable information 
about the potential material requriements to support a fuel cycle for 
advanced reactors. However, they are limited in scope and provide little 
information about how fuel cycle parameters other than the reactors 
deployed and energy demand affect \gls{HALEU} and material needs. 

The proposed work for the remainder of this investigation builds upon 
the already completed studies of the once-through scenarios to 
better understand the impacts of deploying 
\gls{HALEU}-fueled reactors. The next step is to model the transition to 
the Xe-100, \gls{MMR}, and VOYGR in a fuel cycle with recycling to compare 
differences between fuel cycle options. Following that, this investigation 
will perform sensitivity 
analysis on some of the modeled transitions, and use the results of 
the analysis to identify optimized transitions. Sensitivity analysis 
identifies how strongly the input parameters affect 
the material requirements of the transition, which can then be used to 
identify the most effective parameters to optimize the transition. The final step of the 
proposed work investigates the impacts of impurities in 
\gls{HALEU} from downblended \gls{HEU} on the performance of the 
advanced reactors. This step explores how the \gls{HALEU} production 
method affects reactor performance and if this production method 
should be limited. 