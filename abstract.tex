The US is looking into the deployment of advanced reactors that 
require uranium enriched between 5-20\% $^{235}$U, often referred to 
as \gls{HALEU}. There are no facilities 
in the US to produce \gls{HALEU} in the front end of the fuel cycle, 
prompting questions of how to create supplies of \gls{HALEU}. 
\gls{HALEU} can be produce through two primary methods: downblending 
\gls{HEU} and enriching natural uranium. The amount of \gls{HEU} available 
and impurities present in the \gls{HEU} limit downblending capabilties.
The \gls{SWU} capacity and amount of natural urnaium available limit 
enriching natural uranium capabilities. To understand how the US can 
implement each of these production 
methods and the facilities needed, we can quantify the 
material requirements of transitioning to \gls{HALEU}-fueled 
reactors. 

The purpose of this work is to investigate the effects of deploying 
\gls{HALEU}-fueled reactors in the US through material quantification. 
The work performed so far models the transition from the
current fleet of \glspl{LWR} to different subsets of advanced reactors
in a once-through fuel cycle. 
\gls{HALEU}-fueled reactors considered include the X-energy Xe-100 and 
the \gls{USNC} \gls{MMR}. The NuScale VOYGR is also included to understand 
how the deployment of a non-\gls{HALEU}-fueled reactor in tandem with 
\gls{HALEU}-fueled reactors affects the fuel cycle needs. The 
transitions modeled assume both a no growth and a 1\% 
annual growth in energy demand. Materials of interest include the 
number of reactors built, the mass of enriched uranium, the mass 
of feed uranium to produce the enriched uranium, the \gls{SWU} 
capacity needed, and the amount of \gls{SNF} disposed of. 

The results from analyzing the once-through transition scenarios show 
that each of the material requirements are most dependent on the 
reactors deployed in the scenario and less sensitive to the energy 
demand of the scenario. The number of reactors needed in each scenario 
scales with the power output of the reactors deployed in each scenario 
and the energy demand, 
with the transition to only the \gls{MMR} requiring the most reactors for 
each energy demand modeled. The most resources to produce 
\gls{HALEU} are also required by scenarios that deploy only the \gls{MMR}
because of the large number of reactors needed. Deploying the VOYGR 
alongside the \gls{HALEU}-fueled reactors consistently increases the 
mass of enriched uranium requierd and waste generated compared with the 
scenarios deploying 
only \gls{HALEU}-fueled reactors. The increase is a result of the 
VOYGR requiring more fuel in the core than the other two reactors. 
For the feed uranium and \gls{SWU} capacity needed, there is no consistent 
change when deploying the VOYGR alongside the \gls{HALEU}-fueled reactors. 
The change is dependent on the number of VOYGRs deployed in relation to 
the \gls{HALEU}-fueled reactors. All material metrics for \gls{HALEU} 
(mass of \gls{HALEU}, feed uranium, \gls{SWU} capacity, and \gls{HALEU} 
waste) all decrease when VOYGRs are deployed with \gls{HALEU} reactors
because the VOYGR does not require \gls{HALEU}.

The proposed work for the remainder of this investigation builds upon 
the already completed studes of these material requirements to 
better understand the impacts of deploying 
\gls{HALEU}-fueled reactors. The next step is to model the transition to 
the Xe-100, \gls{MMR}, and VOYGR in a fuel cycle with recycling. This 
step will 
quantify the material requirements for deploying these reactors in 
a different fuel cycle option. The next step is to perform sensitivity 
analysis on some of the modeled transitions, and use the results of 
the analysis to identify optimized transitions. Sensitivity analysis 
identifies how the input parameters affect 
the material requirements of the transition, which can then be used to 
identify parameters to optimize the transition. The final step of the 
proposed work investigates the impacts of impurities in 
\gls{HALEU} from downblended \gls{HEU} on the performance of the 
advanced reactors. This step explores how the \gls{HALEU} production 
method may affect reactor performance and if this production method 
should be limited. 