\subsection{Closed fuel cycles}
\begin{frame}
    \frametitle{What about a different fuel cycle option?}
    What if we had a closed fuel cycle that required \gls{HALEU}?
    \begin{itemize}
        \item How does the fuel cycle option impact the material requirements?
        \item Does this save resources?
    \end{itemize}

\end{frame}

\begin{frame}
    \frametitle{Recycle scenario definitions}
    \begin{table}[ht]
        \centering
        \caption{Summary of the recycle fuel cycle transition scenarios.
        Energy growth is relative to energy from \glspl{LWR} in 2025.}
        \label{tab:scenarios_recycle}
        \begin{tabular}{c l l l}
            \hline
            Scenario & Advanced Reactors & Energy demand & Recycle type\\\hline
            \rowcolor{lightorange}14 & Xe-100, MMR, VOYGR & No growth & Limited \\
            \rowcolor{lightorange}15 & Xe-100, MMR, VOYGR & No growth & Limited, no TRISO\\
            \rowcolor{lightorange}16 & SFR& No growth & Continuous \\
            \rowcolor{lightpink}17 & Xe-100, MMR, VOYGR& 1\% growth & Limited \\
            \rowcolor{lightpink}18 & Xe-100, MMR, VOYGR & 1\% growth & Limited, no TRISO\\
            \rowcolor{lightpink}19 & SFR & 1\% growth & Continuous\\
            \hline
    \end{tabular}
    \end{table}
        %<2-> \tikz[overlay, remember picture]{\draw{draw=red,thick, double, fillopacity=0.2] ($(infrastructure)+(-0.5,0.4)$) rectangle ($(infrastructure)+(6,-0.2)$);}} 
\end{frame}

\begin{frame}
    \frametitle{Recycle scenario definitions}
        \begin{table}[ht]
            \centering
            \caption{Summary of the recycle fuel cycle transition scenarios.
            Energy growth is relative to energy from \glspl{LWR} in 2025.}
            \label{tab:scenarios_recycle}
            \begin{tabular}{c l l l}
                \hline
                Scenario & Advanced Reactors & Energy demand & Recycle type\\\hline
                \rowcolor{lightorange}\marktopleft{a3}14 & Xe-100, MMR, VOYGR & No growth & Limited \\
                \rowcolor{lightorange}15 & Xe-100, MMR, VOYGR & No growth & Limited, no TRISO\\
                \rowcolor{lightorange}16 & SFR& No growth & Continuous \markbottomright{a3}\\
                \rowcolor{lightpink}17 & Xe-100, MMR, VOYGR& 1\% growth & Limited \\
                \rowcolor{lightpink}18 & Xe-100, MMR, VOYGR & 1\% growth & Limited, no TRISO\\
                \rowcolor{lightpink}19 & SFR & 1\% growth & Continuous\\
                \hline
        \end{tabular}
        \end{table}
        %<2-> \tikz[overlay, remember picture]{\draw{draw=red,thick, double, fillopacity=0.2] ($(infrastructure)+(-0.5,0.4)$) rectangle ($(infrastructure)+(6,-0.2)$);}} 
\end{frame}

\begin{frame}
    \frametitle{Recycling needs accurate spent fuel compositions}
    \begin{itemize}
    \item The compositions of spent fuel impact numerous fuel cycle considerations:
    \begin{itemize}
        \item decay heat
        \item criticality safety
        \item amount of plutonium and transuranic elements
    \end{itemize}
    \item The \Cycamore \texttt{Reactor} uses recipes to define spent fuel compositions
    \item Other \Cyclus reactor archetypes require export controlled software 
          or are reactor design specific
\end{itemize}
\end{frame}

\begin{frame}
    \frametitle{OpenMCyclus: an open source coupling with OpenMC}
    \begin{itemize}
        \item Developed a reactor archetype that couples \Cyclus with 
              stand-alone depletion solver in OpenMC
        \item Performs depletion on each assembly for each cycle
        \item Publicly available on GitHub \cite{bachmann_openmcyclus_2023}
        \item Benchmarked against the \Cycamore \texttt{Reactor} in a 
              simple closed fuel cycle
    \end{itemize}
\end{frame}

\begin{frame}
    \frametitle{Benchmark Results (I)}
    \begin{columns}
        \column[t]{4cm}
        \begin{itemize}
            \item Separated plutonium masses differ because of 
                  different depletion methodologies
            \item Temporarily changing OpenMCyclus method to better 
                  match with \Cycamore shows better agreement
        \end{itemize}
        \column[t]{6cm}
        \begin{figure}
            \centering 
            \includegraphics[scale=0.4]{comparison_pu_cumulative_discharge.pdf}
            \caption{Comparison of cumulative separated plutonium.}
        \end{figure}
    \end{columns}
\end{frame}

\begin{frame}
    \frametitle{Benchmark Results (II)}
        \begin{itemize}
            \item Differences in separated plutonium masses 
                  propagate into different fuel receipts 
            \item Spent fuel masses are mostly consistent, except 
                  when a reactor is decommissioned
        \end{itemize}
        \begin{figure}
            \centering
            \begin{subfigure}{0.48\textwidth}
                \includegraphics[width=\linewidth]{comparison_spentuox.pdf}
                \caption{Comparison of Spent UOX fuel discharged.}
            \end{subfigure}
            \hfill
            \begin{subfigure}{0.48\textwidth}
                \includegraphics[width=\linewidth]{comparison_spentmox.pdf}
                \caption{Comparison of Spent MOX fuel discharged.}
            \end{subfigure}
            \caption{Spent fuel transactions in OpenMCyclus/\Cycamore benchmark}
            \label{fig:spentfuel_benchmark}
        \end{figure}

\end{frame}


\begin{frame}
    \frametitle{Limited recycle fuel cycle assumptions}
    \begin{columns}
        
    \column[t]{6cm}
    \vspace{-0.9cm}
    \begin{figure}
    \centering
    \begin{tikzpicture}[node distance=0.85cm]
        \node (mine) [facility] {\tiny Uranium Mine};
        \node (enrichment) [facility, below of=mine]{\tiny Enrichment};
        \node (reactor) [facility, below of=enrichment]{\tiny Reactor};
        \node (adv_reactor) [transition, right of=reactor, xshift=1.3cm]{\tiny Advanced Reactor};
        \node (wetstorage) [facility, below of=reactor]{\tiny Wet Storage};
        \node (drystorage) [facility, below of=wetstorage]{\tiny Dry Storage};
        \node (cooling) [transition, below of=adv_reactor]{\tiny Cooling Pool};
        \node (sinkhlw) [facility, below of=drystorage, xshift=1cm]{\tiny HLW Sink};
        \node (sinkllw) [facility, left of=enrichment, xshift=-1cm]{\tiny LLW Sink};
        \node (separation) [transition, below of=cooling]{\tiny Separations};
        \node (fuelfab) [transition, below of=adv_reactor,xshift=1.3cm]{\tiny Fuel Fab};
        
        \draw [arrow] (mine) --(enrichment);
        \draw [arrow] (enrichment) -- (reactor);
        \draw [arrow] (enrichment) -- (sinkllw);
        \draw [arrow] (enrichment) -| (adv_reactor);
        \draw [arrow] (reactor) -- (wetstorage);
        \draw [arrow] (wetstorage) -- (drystorage);
        \draw [arrow] (drystorage) |- (sinkhlw);
        \draw [arrow] (adv_reactor) -- (cooling);
        \draw [dashed, ->] (cooling) -- (separation);
        \draw [arrow] (separation) -| (fuelfab);
        \draw [arrow] (fuelfab) |- (adv_reactor);
        \draw [arrow] (separation) |- (sinkhlw);
        \draw [arrow] (wetstorage) -- (separation);
        \draw [dashed, ->] (cooling) -| (sinkhlw);

        \end{tikzpicture}
    \caption{Fuel cycle facilities and material flow between facilities for the recycling 
    scenarios.}
    \label{fig:limited_fuel_cycle}
\end{figure}

        \column[t]{4.5cm}
        Model transitions using \Cyclus \cite{huff_fundamental_2016}
        \begin{itemize}
            \item Use the same deployment schedule as Scenarios 7, 14
            \item<2-> Modeled Xe-100 and VOYGR with OpenMCyclus
            \item<2-> Calculate MOX compositions based on plutonium
                      equivalence
            \item<3-> Assume natural uranium is enriched to produce all 
                  fuel
        \end{itemize}

\end{columns}
\end{frame}

\begin{frame}
    \frametitle{Continuous recycle fuel cycle assumptions}
    \begin{columns}
        
    \column[t]{6cm}
    \vspace{-0.9cm}
    \begin{figure}
    \centering
    \begin{tikzpicture}[node distance=0.8cm]
        \node (mine) [facility] {\tiny Uranium Mine};
        \node (enrichment) [facility, below of=mine]{\tiny Enrichment};
        \node (reactor) [facility, below of=enrichment]{\tiny Reactor};
        \node (wetstorage) [facility, below of=reactor]{\tiny Wet Storage};
        \node (sinkhlw) [facility, below of=wetstorage, xshift=1cm]{\tiny Repository};
        \node (sinkllw) [facility, left of=enrichment, xshift=-0.5cm]{\tiny Tails Sink};
        \node (separation) [transition, right of=wetstorage,xshift=1cm]{\tiny Separations};
        \node (fuelfab) [transition, above of=separation]{\tiny Fuel Fab};
        \node (fr_enrichment) [transition, right of=enrichment,xshift=2.5cm]{\tiny FR Enrichment};
        \node (sfr) [transition, below of=fr_enrichment]{\tiny Fast Reactor};
        
        \draw [arrow] (mine) -- (enrichment);
        \draw [arrow] (mine) -| (fr_enrichment);
        \draw [arrow] (enrichment) -- (reactor);
        \draw [arrow] (enrichment) -- (sinkllw);
        \draw [arrow] (reactor) -- (wetstorage);
        \draw [arrow] (wetstorage) |- (sinkhlw);
        \draw [arrow] (separation) -- (fuelfab);
        \draw [arrow] (separation) -- (sinkhlw);
        \draw [arrow] (fuelfab) -- (sfr);
        \draw [dashed, ->] (wetstorage) -- (separation);
        \draw [arrow] (sfr) |- (separation);
        \draw [arrow] (fr_enrichment) -- (sfr);

        \end{tikzpicture}
    \caption{Fuel cycle facilities and material flow between facilities for 
    the continuous recycling scenarios.}
    \label{fig:continuous_fuel_cycle}
\end{figure}

        \column[t]{4.5cm}
        Model transitions using \Cyclus \cite{huff_fundamental_2016}
        \begin{itemize}
            \item Introduce a \gls{SFR} for transition, modeled through 
                  OpenMCyclus
            \item<2-> Use the same deployment scheme for the \gls{SFR}
            \item<2-> Calculate HALEU composition based on plutonium
                      equivalence
            \item<3-> Assume natural uranium is enriched to produce all 
                  fuel
        \end{itemize}

\end{columns}
\end{frame}

\begin{frame}
    \frametitle{Advanced reactors}
    \vspace{-0.2cm}
    \begingroup
        \renewcommand{\arraystretch}{1.5}
        \begin{table}
            \centering
            \begin{threeparttable}
        
            \caption{Fast reactor design specification.}
            \label{tab:fast_rx}
            \begin{tabular}{l l}
                \hline
                Design Criteria & PRISM \cite{triplett_prism:_2012,fichtlscherer_assessing_2019}\\
                \hline
                Reactor type & Sodium Fast Reactor\\
                Power Output (MWth) & 840 \\
                Capacity Factor & 90\%\tnote{1} \\
                Enrichment (wt\% fissile Pu) &  11.3/13.5\\
                Cycle Length (yrs) & 1 \\
                Number of cycles &  4\\
                Fuel form &  Metallic \\
                Discharge fuel burnup (GWd/MTU) & 87.51 \\
                Reactor Lifetime (yrs)&  60\\
                \hline
            \end{tabular}
            \begin{tablenotes}
                \item [1] Assumed value
            \end{tablenotes}
        \end{threeparttable}
        \end{table} 
    \endgroup
\end{frame}
