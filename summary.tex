\section{Summary}
The purpose of this work is to investigate the impacts of 
deploying \gls{HALEU}-fueled advanced reactors in the United States. 
The literature 
review summarized different \gls{NFC} options, tools used to perform 
\gls{NFC} analysis and sensitivity analysis, 
the current understanding of the need for \gls{HALEU}, and known 
effects of using \gls{HALEU} in reactors on the nuclear fuel cycle 
and reactor performance. The methodology chapter 
described how models of advanced reactors and once-through fuel cycle 
simulations were constructed, leading to the results presented in the 
fourth chapter. 

The work presented models the transition from the current fleet of 
\glspl{LWR} in the US to various combinations of advanced reactors, 
specifically the X-energy Xe-100, \gls{USNC} \gls{MMR}, and the 
NuScale VOYGR, assuming either a no growth or 1\% annual growth in 
energy demand. The results from these scenarios include the material 
requirements of the transitions, specifically the number of reactors 
deployed to meet a prescribed energy demand, the 
uranium resources, the \gls{SWU} 
capacity, and the waste produced. The number of reactors required 
in each transition scenario scales based on the power output of the 
reactors and the energy demand of the scenario. The uranium resources, 
\gls{SWU} capacity and waste discharged are 
affected more by the reactors deployed than the energy demand. Deploying 
the 
NuScale VOYGR (an advanced reactor that does not require \gls{HALEU}) 
in tandem with the X-energy Xe-100 or \gls{USNC} \gls{MMR} (reactors 
that require \gls{HALEU}) reduces the need for resources related to 
\gls{HALEU}, but does not necessarily decrease the total amount of 
enriched uranium required. 
The mass of enriched uranium required and the mass of waste discharged 
increase when deploying the VOYGR alongside the other reactors, but the 
feed uranium mass and \gls{SWU} capacity are comparable to when the VOYGR 
is not deployed. These results 
highlight how the mass for a core loading and the 
uranium assay of the fuel for each reactor type affect the material 
requirements. The 
quantification and understanding of these material needs aid in 
establishing fuel cycle facilities and resources to meet the needs of 
the transition. 

\subsection{Corrected deployment methodology}
To correct the flaw in the advanced reactor deployment methodology, 
the deployment scheme 
will be altered to match energy produced to the energy demand, instead 
of matching installed capacity to energy demand. Making this change 
involves updates to the capacity factor of each reactor type to 
accurately capture the energy each reactor type will produce. 

The \gls{LWR} fleet is currently modeled with a 94.7\% capacity factor, 
which is higher than historic data of actual capacity factors 
\cite{us_eia_monthly_2022}, as discussed in Section \ref{sec:scenario1}.
In all updated and future work 
I will use a capacity factor of 92.66\% for the \glspl{LWR}, which 
is an average of the reported capacity factors for the \gls{LWR} fleet 
over the last 5 years \cite{noauthor_electric_2022}. The NuScale VOYGR 
is also currently modeled with 
a 94.7\% capacity factor, which will be updated to 95\%, based on 
NuScale's projected capacity factor for the reactor 
\cite{nuscale_technology_nodate}. The Xe-100 and \gls{MMR} are currently 
modeled as a 100\% capacity factor, because they are each modeled with a 
zero-month refueling duration. The Xe-100 will be modified to a 95\% 
capacity factor, based on estimates from X-energy 
\cite{xenergy_reactor_nodate}. The \gls{MMR} will continue to be modeled 
with a 100\% capacity factor because the difference between 
5 MWe and 4.75 MWe (power output when using a 95\% capacity factor) is 
not expected to have any meaningful effect on the results of this work. 

To account for each of these capacity factors, two changes need to be 
made 
to the definition of each reactor type. First, the power output of each 
reactor type is multiplied by the appropriate capacity factor. This 
changes the power output of the Xe-100 to 76 MWe and the power output of 
the VOYGR to 73 MWe. The power output of each of the \glspl{LWR} is 
decreased to 92.66\% of the power listed in the \gls{PRIS} database. 
Second, the VOYGR and \glspl{LWR} are altered to have a refueling 
duration of zero months and their operating cycle is extended by one 
month. This alteration ensures that only the specified capacity factor 
is applied
to each of these reactors, and limitations in the modeling capabilities 
do not affect the capacity factor or introduce artificial changes to the 
reactor capacity factor. By extending the operating 
cycle by one month, the reactors still receive fuel on the same schedule 
as when the refueling outages were explicitly modeled. 

These changes to the modeling methodology are expected to affect each of 
the results of this work. First, the energy produced by the \gls{LWR} 
fleet in 
2025 is expected to change, because they are now modeled to have a 
smaller maximum power output. 
This change in the results will affect the energy demand of each of the 
transition scenarios, thus affecting the number of advanced reactors 
deployed and the material requirements of the advanced reactors. This 
new methodology does not affect the amount of fuel and resources required 
by each type of reactor or when they require them. 


\section{Proposed Work}
The preliminary work accomplishes part of the first objective of this work: 
quantify potential material requirements for the transition from 
\glspl{LWR}
to advanced reactors in open and closed fuel cycles. The results identify 
potential material needs for transitioning to various combinations of 
advanced reactors assuming different energy demands. However, this work 
is very limited in scope, and does not investigate the material requirements 
of these transitions with a closed fuel cycle or how model parameters 
other 
than the reactor types and energy demand affect the results. Additionally, 
this work only considers the enrichment of natural uranium to produce 
\gls{HALEU}. 

The proposed work expands upon the once-through transition analysis in 
evaluating the effects of deploying \gls{HALEU}-fueled reactors in the 
United States. 
To investigate how changing the fuel cycle to a closed fuel affects the 
material requirements, the transition to advanced reactors in a closed fuel 
cycle will be modeled. This step will complete the first 
objective of this work. These scenarios and the once-through scenarios 
investigate how the reactors, energy demand, and fuel cycle type affect 
the material requirements. However, previous fuel cycle analysis has 
shown that other model parameters, such as the transition start time 
and lifetime of existing reactors, will also affect the material 
requirements. Therefore, 
the next stage of proposed work will perform sensitivity analysis on 
select once-through and recycle fuel cycles modeled. The results of the 
sensitivity 
analysis will then used to develop optimized transition scenarios based 
on minimizing specific material requirements. This section of work will 
complete the second objective of this work. Finally, downblending 
\gls{HEU} is a viable 
method to produce \gls{HALEU}, but impurities in the \gls{HALEU} may limit 
the amount of downblended \gls{HEU} that can be present in a reactor core 
at once. To investigate potential limitations on using downblended 
\gls{HEU} in the advanced reactors, this work will perform depletion 
calculations and neutronics analysis of reactor performance using 
\gls{HALEU} produced by 
downblending \gls{HEU}. This step will complete the third and final
objective of this work. 

\subsection{Recycle Fuel Cycles}

\subsection{Sensitivity Analysis and Optimization}


The results of the \gls{OAT} analysis will be reported as figures 
to describe general trends and statistics (maximum, minimum, and 
average) to quantify the effects of the input parameter. 
The results 
of the synergistic analysis will be reported using surface plots show 
the relationship 
between the inputs and the output metric, similar to results presented in 
\cite{chee_sensitivity_2019} and \cite{passerini_systematic_2014}.
The results of the global analysis will be reported using Sobol' 
indices, which have been used in \cite{richards_application_2021}. 

The scenarios will then be optimized based on specific goals, to be 
identified after completion of the sensitivity analysis. 
Potential 
optimization problems include minimizing the \gls{HALEU} mass required 
or minimizing \gls{SNF} produced in Scenarios 7 and 14. Two 
single-objective and one multi-objective optimization problems will be 
constructed for this work.  

\subsection{Neutronics Analysis}
