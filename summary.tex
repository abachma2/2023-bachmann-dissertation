\section{Summary}
The purpose of this work is to investigate the impacts of 
deploying \gls{HALEU}-fueled advanced reactors in the US. The literature 
review discussed different \gls{NFC} options, tools used to perform 
\gls{NFC} analysis and sensitivity analysis, 
the current understanding of the need for \gls{HALEU}, and neutronics 
effects of using \gls{HALEU} in reactors. 

The work thus far models the transition from the current fleet of 
\glspl{LWR} in the US to various combinations of advanced reactors, 
specifically the X-energy Xe-100, \gls{USNC} \gls{MMR}, and the 
NuScale VOYGR, assuming either a no growth or 1\% annual growth in 
energy demand. The results from these scenarios include the material 
requirements of the transitions, specifically the number of reactors, the 
uranium resources, the \gls{SWU} 
capacity, and the waste produced. The number of reactors required 
in each transition scenario scales based on the power output of the 
reactors and the energy demand of the scenario. The other resources are 
affected more by the reactors deployed than the energy demand. Deploying the 
NuScale VOYGR (an advanced reactor that does not require \gls{HALEU}) 
in tandum with the X-energy Xe-100 or \gls{USNC} \gls{MMR} (reactors 
that require \gls{HALEU}) reduces the need for resources related to 
\gls{HALEU}, but not necessarily the total amount of the resource required. 
The mass of enriched uranium required and the mass of waste discharged 
increase when deploying the VOYGR alongside the other reactors, but the 
feed uranium mass and \gls{SWU} capacity sometimes decreased. These results 
highlight how the mass for a core loading and the 
uranium assay of the fuel for each reactor type affect the material 
requirements. The 
quantification and understanding of these material needs aid in 
establishing fuel cycle facility and resources to meet the needs of 
the transition. 

\section{Proposed Work}
The remaining proposed work expands upon the work already performed in 
evaluating the effects of deploying \gls{HALEU}-fueled reactors in the US. 
The proposed work considers the transition to these reactors for a 
closed fuel cycle to investigate how this using recycling in the fuel 
cycle affects the material requirements. The proposed work also performs 
sensitivity analysis on the transition for both types of fuel cycle 
(open and closed) to determine the amount of impact some of the modeling 
decisions have on the material requirements. The results of the sensitivity 
analysis are then used to develop optimized transition scenarios based on 
some of the material requirements. Finally, the effects of known impurities 
from downblending \gls{HEU} to create \gls{HALEU} on the performance of 
the advanced reactors will be investigated. 
The work performed already and the first two 
parts of the proposed work focus on how the reactors deployed affect 
the \gls{NFC}. The final part of the proposed work investigates how the 
fuel cycle affects the reactors and their performance. 

\subsection{Recycle Fuel Cycles}
The once-through fuel cycles investigated will be modified to include recycling 
to investigate how this type of \gls{NFC} affects the material requirements 
and how the recycling scheme impacts them as well. 
The scenarios investigated for this section of analysis vary by the 
energy demand of the scenario and the recycling scheme. Energy demands vary 
between a no growth and a 1\% growth model, the same as the variation in this 
parameter in the once-through fuel cycle models. Variations in the recycling 
scheme include either a limited or a continuous recycle, similar to the 
recycling schemes considered in the \gls{ES} \cite{wigeland_nuclear_2014}. 
All of the scenarios 
with recycling will model the transition from the current fleet of \glspl{LWR} 
to the X-energy Xe-100, \gls{USNC} \gls{MMR}, and the NuScale VOYGR, the same 
as in scenarios 7 and 13. 

\begin{table}[ht]
    \centering
    \caption{Summary of the recycle fuel cycle transition scenarios.}
    \label{tab:scenarios_recycle}
    \begin{tabular}{l l l}
            \hline
            Scenario number & Energy growth model & Recycle scheme\\\hline
            14 & No growth & Limited \\
            15 & No growth & Continuous \\
            16 & 1\% growth & Limited \\
            17 & 1\% growth & Continuous \\
            \hline
    \end{tabular}
\end{table}

Figure \ref{fig:limited_recycle_flow} shows the fuel cycle and material flows 
for the scenarios with limited recycling (Scenarios 14 and 16). This fuel 
cycle assumes that \gls{SNF}
from the \glspl{LWR} and advanced reactors can undergo separation and be 
recycled. A separations facility is defined to be deployed 5 years before 
the transition 
begins (i.e. in 2020), as this strategy is commonly used for modeling 
transitions with recycling \cite{passerini_systematic_2014,richards_application_2021}
to ensure that enough fuel can be spearated and 
processed in time for use in advanced reactors.

\begin{figure}
    \centering
    \begin{tikzpicture}[node distance=1.5cm]
        \node (mine) [facility, text width=1cm] {Uranium Mine};
        \node (enrichment) [facility, below of=mine]{Enrichment};
        \node (reactor) [facility, below of=enrichment]{LWR};
        \node (mmr) [transition, right of=reactor, xshift=1.5cm]{MMR};
        \node (xe100) [transition, right of=mmr, xshift=1.5cm]{Xe-100};
        \node (voygr) [transition, right of=xe100, xshift=1.5cm]{VOYGR};
        \node (wetstorage) [facility, below of=reactor, text width=1cm]{Cooling Pool};
        \node (drystorage) [facility, below of=wetstorage, text width=1.5cm]{Dry Storage};
        \node (cooling) [transition, below of=mmr, xshift=1.5cm, text width=1cm]{Cooling Pool};
        \node (sinkhlw) [facility, below of=drystorage, xshift=2.5cm, yshift=-1cm]{Repository};
        \node (sinkllw) [facility, left of=enrichment, xshift=-1.5cm, text width=1cm]{Tails Sink};
        \node (separation) [transition, below of=cooling, yshift=-1cm]{Separations};
        \node (mox_fab) [transition, below of=voygr,xshift=2cm, text width=1cm]{MOX Fuel Fab};
        \node (mox_cooling) [transition, below of=xe100, xshift=1.5cm, yshift=-1.5cm]{MOX Cooling};
        
        \draw [arrow] (mine) -- node[anchor=west]{Natural uranium} (enrichment);
        \draw [arrow] (enrichment) -- node[anchor=east]{Fresh UOX}(reactor);
        \draw [arrow] (enrichment) -- node[anchor=south]{Tails}(sinkllw);
        \draw [arrow] (enrichment) -| (mmr);
        \draw [arrow] (enrichment) -| node[anchor=south]{Fresh UOX}(xe100);
        \draw [arrow] (enrichment) -| (voygr);
        \draw [arrow] (reactor) -- node[anchor=east]{Spent UOX}(wetstorage);
        \draw [arrow] (wetstorage) -- node[anchor=east]{Cool Spent UOX}(drystorage);
        \draw [arrow] (drystorage) |- node[anchor=east]{Casked Spent UOX}(sinkhlw);
        \draw [arrow] (mmr) |- node[anchor=east]{Spent UOX}(cooling);
        \draw [arrow] (xe100) |- (cooling);
        \draw [arrow] (voygr) |- node[anchor=south]{Spent UOX}(cooling);
        \draw [arrow] (xe100) -| (mox_cooling);
        \draw [arrow] (voygr) -| node[anchor=south, text width=1cm, pos=0.25]{Spent MOX}(mox_cooling);
        \draw [arrow] (cooling) -- node[anchor=west, text width=1.5cm]{Cooled Spent UOX}(separation);
        \draw [arrow] (separation) -| node[anchor=south,text width=1.5cm, pos=0.4]{Separated fissile material}(mox_fab);
        \draw [arrow] (separation) -| node[anchor=south, text width=1.5cm]{Fission Products}(sinkhlw);
        \draw [arrow] (mine) -| node[anchor=south, pos=0.4]{Natural uranium} (mox_fab);
        \draw [arrow] (mox_cooling) |- node[anchor=south]{Cool Spent MOX}(sinkhlw);
        \draw [arrow] (mox_fab.east) - ++(5mm,0) |- node[anchor=south, pos=0.5, text width=1.5cm]{Fresh MOX}(voygr.east);
        %\draw [arrow] (mox_fab.east) -| ++(5mm,0) -| (xe100.north);
        \end{tikzpicture}
    \caption{Fuel cycle facilities and material flow between facilities for modeling the transition 
    to advanced reactors with a limited recycle fuel cycle. The separations facility
    is deployed in 2020, five years before the transition. Before 2020 a once-through 
    fuel cycle is used with the facilities in blue.}
    \label{fig:limited_recycle_flow}
\end{figure}

Figure \ref{fig:continuous_recycle_flow} shows the facilities and material 
flow for the continuous recycle fuel cycle. Continuous recycle 
requires a fast reactor, and all of the reactors considered in this 
work have a thermal neutron spectrum. Therefore, a fast reactor facility 
must be added into the transition. To accomplish this, two different 
transitions are included in these scenarios: the transition from 
\glspl{LWR} to the advanced reactors and the transition from the 
advanced reactors to the fast reactor. The first transition will begin 
in 2025, and the second transition will begin in 2060, and the deployment 
rate of each reactor type will be based on the energy demand of the 
scenario. 
The fast reactor will be modeled 
based on the fast reactor technology used in the \gls{ES} 
\cite{wigeland_nuclear_2014} to provide a generic representation of 
a fast reactor and not any specific design of fast reactor technology. 


\begin{figure}
    \centering
    \begin{tikzpicture}[node distance=1.5cm]
        \node (mine) [facility] {Uranium Mine};
        \node (enrichment) [facility, below of=mine]{Enrichment};
        \node (reactor) [facility, below of=enrichment]{LWR};
        \node (adv_reactor) [transition, right of=reactor, xshift=3cm]{Advanced Reactor};
        \node (wetstorage) [facility, below of=reactor]{Wet Storage};
        \node (drystorage) [facility, below of=wetstorage]{Dry Storage};
        \node (cooling) [transition, below of=adv_reactor]{Cooling Pool};
        \node (sinkhlw) [facility, below of=drystorage, xshift=2.5cm]{Repository};
        \node (sinkllw) [facility, left of=enrichment, xshift=-3cm]{Tails sink};
        \node (separation) [transition, below of=cooling]{Separations};
        \node (fuelfab) [transition, below of=adv_reactor,xshift=3cm]{Fuel Fab};
        \node (sfr) [transition, right of=adv_reactor, xshift=3.5cm]{Fast Reactor};

        \draw [arrow] (mine) -- node[anchor=east]{Natural U} (enrichment);
        \draw [arrow] (enrichment) -- node[anchor=east]{Enriched U}(reactor);
        \draw [arrow] (enrichment) -- node[anchor=south]{Tails}(sinkllw);
        \draw [arrow] (enrichment) -| node[anchor=west]{Fresh Fuel}(adv_reactor);
        \draw [arrow] (reactor) -- node[anchor=east]{Spent UOX}(wetstorage);
        \draw [arrow] (wetstorage) -- node[anchor=east]{Cool Spent UOX}(drystorage);
        \draw [arrow] (drystorage) |- node[anchor=east]{Casked Spent UOX}(sinkhlw);
        \draw [arrow] (adv_reactor) -- node[anchor=west]{Spent Fuel}(cooling);
        \draw [arrow] (cooling) -- node[anchor=west]{Cooled Spent Fuel}(separation);
        \draw [arrow] (separation) -| node[anchor=north, text width=1.5cm]{Separated fissile material}(fuelfab);
        \draw [arrow] (fuelfab) |- node[anchor=south, text width=1.5cm]{Reprocessed fuel}(sfr);
        \draw [arrow] (fuelfab) |- (adv_reactor);
        \draw [arrow] (separation) |- node[pos= 0.3, anchor=west, text width = 1.5cm]{Fission products}(sinkhlw);
        \draw [arrow] (wetstorage) -- node[sloped, anchor=south]{Cool Spent UOX}(separation);
        \draw [arrow] (sfr) |- node[anchor=west]{Waste}(sinkhlw);
        \draw [arrow] (mine) -| node[anchor=south]{Natural uranium}(sfr);

        \end{tikzpicture}
    \caption{Fuel cycle facilities and material flow between facilities for modeling the transition 
    to advanced reactors with a continuous recycle fuel cycle. Before 2020 a once-through 
    fuel cycle is used with the facilities in blue. Facilities in red are deployed 
    in 2025, and represent all of the facilities and material flows in red in 
    Figure \ref{fig:limited_recycle_flow}.}
    \label{fig:continuous_recycle_flow}
\end{figure}

\subsection{Sensitivity Analysis and Optimization}
The sensitivity analysis identifies how the variation of select
parameters affect select performance 
metrics of the fuel cycle and the combined effect of multiple parameters on 
an output metric. This information is then used to design 
an optimized transition scenario. 

To perform the sensitivity analysis and 
transition optimization \Cyclus will be coupled to Dakota \cite{adams_dakota_2019},
and Dakota will be the driver for the analysis. This coupling will be based on 
the scripts in the \texttt{dcwrapper} GitHub repository \cite{chee_arfc/dcwrapper_2019}.
Three different types 
of sensitivity analysis will be performed: \gls{OAT}, synergystic, 
and global. The analysis will be performed on Scenario 7 and Scenario 
14 to provide insight into how the input parameters affect the output 
metrics for both a once-through and recycle fuel cycle. 
The input parameters to be varied include the transition 
start time, the market share of each type of advanced reactor, and 
the \gls{LWR} lifetime. The transition start time will range from January 
2025 to January 2040 in three-month intervals, but the same energy demand 
will be specified for all perturbations (89.45 GWe-yr). The market share 
variable will be repeated three times, for varying the market share of each 
type of advanced reactor) with market share percentages ranging from 0-50\% 
in increments of 5\%. Finally, the \gls{LWR} lifetimes will be varied based 
on the percent of the fleet that operate for 60 and 80 years. This 
variable will vary between 0-50\% of the \gls{LWR} fleet operating for 80 
years. The \glspl{LWR} do not all start operation at the same time, so the 
selection of the \glspl{LWR} that operate for 80 years will impact the results, 
even if the number is the same. Therefore, reactors will be selected in 
descending order of power output, reflecting the greater likelihood of 
larger units receiving a license extension. Previous sensitivity analysis of 
fuel cycle transitions considered the impact of the transition start time 
and the \gls{LWR} lifetimes \cite{chee_sensitivity_2019,feng_sensitivity_2020}.

The output metrics of 
interest for this analysis include the amount of waste generated that 
must be sent to a repository, the mass of enriched uranium, the mass of 
\gls{HALEU}, and 
the amount of \gls{SWU} capacity required to produce all enriched uranium, the 
\gls{SWU} capacity required to produce \gls{HALEU}, and the feed uranium 
required to produce \gls{HALEU}. Each metric will be evaluated based on the 
cumulative sum required, starting at the transition start time. Previous 
sensitivity analysis of fuel cycle transitions considered the waste 
discharged, \gls{SWU} 
capacity required, and natural uranium requirements
\cite{richards_application_2021,feng_sensitivity_2020} 

The results of the one-at-a-time analysis will be reported as figures 
to describe general trends and statistics (maximum, minimu, average, 
standard deviation) to quantify the effects of the input parameter. 
The results 
of the synergystic analysis will be reported using surface plots show 
the relationship 
between the inputs and the output metric, similar to the plots shown in 
\cite{chee_sensitivity_2019} and \cite{passerini_systematic_2014}.
The results of the global analysis will be reported using Sobol' 
indices, which have been used in \cite{richards_application_2021}. 

\subsection{Neutronics Analysis}
Recovery and downblending of \gls{HALEU} to produce \gls{HALEU} means that 
the fuel is likely to contain impurities present in the 
\gls{HEU}. Known impurities in potential \gls{HEU}
supplies to create \gls{HALEU} include $^{232}$U and $^{236}$U
\cite{vaden_isotopic_2018,nelson_foreign_2010},  
which are parasitic neutron absorbers. This work models the use of 
impure \gls{HALEU} fuel in a reactor and compares it with \gls{HALEU} fuel 
without these impurities to understand how the impurities affect 
the performance of the reactor. 

Full core models of the Xe-100 and \gls{MMR} will be created in Serpent 
\cite{leppanen_serpent_2014} to model the neutronics of each reactor 
design. The fuel composition for each core will be varied to provide 
a comparison of pure (containing only $^{235}$U and ${238}$U) and only 
impure fuel.
Material compositions of impure fuel will be modeled based on both the 
known compositions of \gls{HALEU} from the \gls{EBR} stockpile 
\cite{vaden_isotopic_2018} and the Y-12 stockpile \cite{nelson_foreign_2010}.
Each core will be modeled three times, once for each fuel composition considered.
By modeling the core composition to be entirely composed of impure fuel 
this work reports the most extreme effect that these impurities can 
have on reactor performance.  

Results of each simulation to be compared include the neutron flux at 
\gls{BOL}, mid-cycle, and \gls{EOL}, $k_{eff}$ as a function of burnup, 
fuel temperature reactivity coefficient, and 
$\beta_{eff}$. Each parameter impacts the performance of 
the reactor, such as the materials degradation rate, amount of burnable 
poisons required, control rod worth, and the cycle time. Investigating 
each of these results helps to determine if the impurities potentially 
present in \gls{HALEU} will prevent any of the design criteria of the 
reactors from being met. 