\subsection{Conclusions}
\begin{frame}
      \frametitle{Conclusions}
      \begin{itemize}
        \item This work investigates the impacts of deploying \gls{HALEU}-fueled 
              reactors on the nuclear fuel cycle.
        \item<2-> The material requirements of the transitions modeled are governed 
              by the design characteristics of the reactors deployed
        \item<2-> Closing the fuel cycle decreases material needs, but the 
              decrease is governed by the recycling scheme and the 
              material available for reprocessing
        \item<3-> Sensitivity analysis highlighted how the 
              advanced reactor characteristics affect the material requirements, and 
              the importance of the Xe-100 burnup
        \item<3-> Found transitions to minimize \gls{HALEU} \gls{SWU} 
              and \gls{SNF} mass, but other algorithms should 
              be used when using a linear constraint
        \item<4-> The impurities from downblending \gls{HEU} affect 
              reactor parameters, but won't necessarily prevent key design 
              parameters from being met
      \end{itemize}
\end{frame}

\begin{frame}
      \frametitle{A methodology for comprehensive fuel cycle analysis}
      \begin{itemize}
            \item Demonstrate how to expand transition analysis with 
                  sensitivity analysis
            \item Develop a reactor-agnostic archetype to dynamically model depletion 
                  in \Cyclus, without needing an export-controlled code
            \item<2-> Provide a detailed insight on how parameters and 
                  decisions affect fuel cycle needs
      \end{itemize}
\end{frame}

\subsection{Limitations \& Future work}
\begin{frame}
      \frametitle{Limitations and Future Work}
      \begin{itemize}
            \item Transition analysis provided a macroscopic view of 
                  material needs, focused on fuels
            \begin{itemize}
                  \item<2-> Model the needs of non-fuel materials, 
                        like reactor-grade graphite
                  \item<2-> Determine and model facility capacities
                  \item<2-> Account for processing time 
            \end{itemize}
            \item<3-> Ignores other externalities, like nonproliferation
                    safeguards
            \begin{itemize}
                  \item<4-> Incorporate potential safeguards measures into models
            \end{itemize}
            \item<5-> Expand analysis on impurities in \gls{HALEU}
            \begin{itemize}
                  \item<6-> Consider power-peaking factors
                  \item<6-> Model burnable poisons and control rods 
                  \item<6-> Model core in non-isothermal state
            \end{itemize}
      \end{itemize} 
\end{frame}