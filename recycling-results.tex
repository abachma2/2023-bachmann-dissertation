This chapter reports the results for the recycle fuel cycle 
scenarios (Scenarios 14-19) described 
in Section \ref{sec:sec:recycle-methods}. The primary results considered 
for these fuel cycle transitions are the uranium resources needed, 
the \gls{SWU} capacity requried, the separated plutonium masses, 
and the mass of material disposed of. This chapter does not focus 
as much on the number of reactors or the energy supplied by 
the reactors because most of the scenarios use the same 
deployment scheme as Scenarios 7 or 14 (depending on 
the energy demand curve of the scenario). For the two 
scenarios that deploy the \gls{SFR} instead of the other 
advanced reactors (Scenarios 16 and 19), the maximum number 
of \glspl{SFR} deployed in each scenario is 312 and 595, 
respectively. 
These scenarios require far fewer reactors than the other scenarios 
because the \gls{SFR} has a larger power output than the other 
advanced reactors (311 MWe compared with 80 MWe for the Xe-100).

\section{Uranium resources}

\subsection{No growth scenarios}

\subsection{1\% growth scenarios}

\section{SWU capacity}

\subsection{No growth senarios}

\subsection{1\% growth scenarios}

\section{Separated plutonium}

\subsection{No growth scenarios}

\subsection{1\% growth scenarios}

\section{Waste}

\subsection{No growth scenarios}

\subsection{1\% growth scenarios}