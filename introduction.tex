\section{Motivation}
In the US, the current fleet of nuclear reactors is comprised 
of two \gls{LWR} designs: \glspl{PWR} and \glspl{BWR}. 
Both designs use uranium dioxide pellets as fuel, with the uranium 
enriched to no more than 5\% $^{235}$U and supply about 1000 MW of 
power. These types of reactors 
have commercially operated in the US since 1957. This fleet of 
reactors supplied about 19\% of all energy and over 50\%  
of all carbon-free energy in the US in 2021 
\cite{us_energy_information_administration_electricity_2022}, making nuclear 
energy the third largest producer of energy and the largest 
producer of carbon-free energy in the United States. Because of nuclear 
power's ability to produce large-scale carbon-free energy, it is 
expected to play a role in meeting carbon emission and climate change 
goals \cite{nea_meeting_2022}. The \glspl{LWR} 
currently deployed in the US have license expiration dates within 
the next 35 years, with the last license expiring in 2055 
\cite{nuclear_energy_institute_us_2021}. Therefore, if nuclear energy is 
to continue to produce energy in the US and assist in meeting carbon 
emission goals after 2050, we must extend current reactor licenses or 
build new reactors. 

Multiple countries around the world are building new \glspl{LWR}
\cite{world_nuclear_association_plans_2022}, and developing
new reactor designs \cite{hussain_advances_2018}, often called 
advanced reactors, to 
replace or expand the current fleet of reactors. These advanced reactors 
cover a large swath of design space, with wider ranges in: energy output, 
fuel form, and cycle length. The variations in reactor designs allow 
advanced reactors to achieve higher fuel burnup, improved safety 
performance, and better economic competitiveness than the \gls{LWR} fleet. 
In the US, the \gls{DOE} established the \gls{ARDP} 
\cite{us_department_of_energy_office_of_nuclear_energy_advanced_nodate}
to ``speed the demonstration of advanced reactors'' 
\cite{us_department_of_energy_office_of_nuclear_energy_advanced_nodate}
by developing a cost-sharing program with private companies. The goal of 
\gls{ARDP} is to leverage this cost-sharing program to build 
first-of-a-kind advanced reactors to assist in their licensing and 
understanding challenges and opportunities in the construction process. 
The reactors built through this program are planned to be complete by the 
late 2020's.

One important question that has arisen from the \gls{ARDP} is how to
develop supply chains to support the nuclear fuel cycle of 
advanced reactors. One design parameter that is different from \glspl{LWR} 
and almost every advanced reactor is the fuel form. \glspl{LWR} require 
a ceramic uranium dioxide fuel, while advanced reactors require fuel 
forms such as \gls{TRISO} fuel kernels with uranium oxycarbide, molten 
salts, and metallic alloy fuels \cite{hussain_advances_2018}.
Additionally, many advanced reactor designs require  
\acrfull{HALEU}, which is uranium enriched between 5\% and 20\% $^{235}$U,
compared with the 3-5\% $^{235}$U that is required for \gls{LWR} 
fuel. There is no commercial supply of \gls{HALEU} in the US that can 
supply fuel for the \gls{ARDP} projects or for a potential future fleet of 
advanced reactors. Therefore, the \gls{DOE} is investigating how to develop 
a supply chain and \gls{NFC} for uranium enriched to this 
increased level \cite{regalbuto_addressing_2020,dixon_estimated_2022}. 

There are two methods to produce \gls{HALEU}: enrich natural uranium up to 
the required level
or downblend \acrfull{HEU} to the desired enrichment level. The current US
nuclear commercial fuel cycle relies on enrichment of natural uranium 
to create fuel for \glspl{LWR}, but does not have the capability to enrich
natural uranium to produce \gls{HALEU} \cite{nuclear_energy_institute_addressing_2018}.  
The amount of available natural uranium and facility capacities, such 
as the material throughputs and \acrfull{SWU} capacity, limit 
\gls{HALEU} production through enrichment. If \gls{HEU} were to be 
downblended to supply \gls{HALEU}, \gls{HEU} supplies
considered for downblending comes from three different 
\gls{DOE} sources: the spent \gls{EBR} fuel stockpiles at \gls{INL} 
\cite{patterson_haleu_2019}, stockpiles at \gls{SRS} \cite{regalbuto_addressing_2020}, 
and stockpiles at Y-12 National Security Complex 
\cite{robinson_establishment_2020}. The size of each stockpile limits the amount 
of \gls{HALEU} produced through downblending, and the first two stockpiles 
are capable of producing no more than 20 MTU of \gls{HALEU}
\cite{regalbuto_addressing_2020}.
Additionally, two of the potential stockpiles of \gls{HEU} to downblend  
contain impurities that would affect reactor performance 
\cite{vaden_isotopic_2018,nelson_foreign_2010}.
These impurities may limit the amount of downblended \gls{HEU} that a reactor 
a reactor can use at once. 
Currently, there is only one facility in the US commercially licensed to 
downblend \gls{HEU}, the BWXT Nuclear Fuel Services Inc. facility in 
Erwin, TN, which is expected to have the capacity to downblend 1-2 
MT of \gls{HEU}, producing up to 10 MT of \gls{HALEU} each year \cite{nagley_ha-leu_2020}.
Therefore, meeting \gls{HALEU} demand through downblending may require the 
development and licensing of additional facilities to downblend \gls{HEU} if 
the BWXT facility does not have enough capacity. 

The production of fuels needed for deploying advanced reactors 
is expected to have numerous impacts on the \gls{NFC}, stemming from 
the different materials and enrichment levels needed for advanced reactor
fuels. To assist in understanding the impacts 
from deploying \gls{HALEU}-fueled reactors, modeling the \gls{NFC} 
can provide the material requirements of potential transition scenarios. 
The technology employed in the \gls{NFC} often defines the fuel 
cycle type, and a transition in the nuclear fuel cycle occurs when 
introducing new fuel types or other fuel cycle technologies.  
Modeling  a \gls{NFC}, typically aided by a fuel cycle simulator, 
includes modeling the deployment and decommissioning of facilities in 
the \gls{NFC}, such as mines or reactors, and modeling the materials 
traded between facilities. 
\gls{NFC} simulators have been used to model a variety of \gls{NFC} 
transitions \cite{sunny_transition_2015,bae_fuel_2018,piet_dynamic_2011} 
and quantify the resource requirements of \glspl{NFC} transitions
\cite{bachmann_enrichment_2021}. Therefore, using \gls{NFC} simulators to 
model the transition from the US fleet of \glspl{LWR} to potential 
fleets of advanced reactors can inform \gls{DOE}, researchers, companies, 
and other key stakeholders of potential \gls{HALEU} needs for future 
advanced reactors deployed. Estimates on potential needs can then inform 
ways to develop material supply chains to fuel advanced reactors. 

When modeling the \gls{NFC} to assist in answering questions about 
\gls{HALEU} demand and other resources required to meet the demand, 
there is a large array of input parameters that must be considered: when to start 
a transition, the speed of the 
transition, and the advanced reactors to deploy. Many times, modelers make 
assumptions about transition parameters \cite{sunny_transition_2015, djokic_application_2015}
and others use energy projections to determine transition parameters 
\cite{dixon_estimated_2022}. These are valid methods to determine 
transition parameters, but only considering a select set of parameters may 
not capture all of the possible material demands of a transitions. 
Therefore, fuel cycle modelers often perform sensitivity analysis on 
the transition to better understand a broader range of material 
requirements and how their assumptions affect demand. 
Sensitivity analysis involves 
modeling a fuel cycle with small perturbations in various input 
parameters and analyzing the variance or spread of select output metrics. 
Sensitivity analysis has been used in multiple \gls{NFC} analysis 
\cite{chee_sensitivity_2019,feng_sensitivity_2020,thiolliere_methodology_2018,passerini_sensitivity_2012}
to identify the model parameter or parameters that have the greatest 
impact on specific model outputs or material requirements. Sensitivity 
analysis can also reveal underlying information about a system that is not 
necessarily intuitive, such as how much the modeling methodology 
affects the results. Understanding the 
material requirements of a \gls{NFC} coupled with sensitivity analysis 
aids in optimizing the \gls{NFC} based on given criteria, such as 
minimizing the amount of \gls{HALEU} required. Various 
optimization schemes have been applied to \glspl{NFC} to optimize the fuel 
cycle based on minimizing certain criteria: fuel requirements \cite{kim_selection_1999},
waste production \cite{shwageraus_optimization_2003}, and combinations 
of these metrics in multi-objective 
problems \cite{passerini_systematic_2014}. By combining \gls{NFC} transition 
analysis with sensitivity analysis and optimization,one obtains
a deep understanding of potential \gls{HALEU} demand, potential supply 
chain requirements, and how to alter reactor deployment to aid in developing 
the supply chain. 

\section{Research Goals}
The goal of this work is to investigate the impacts of deploying reactors 
fueled 
by \gls{HALEU} in the United States, including the impacts that the reactors 
have on the \gls{NFC} and the impacts the \gls{NFC} has on the reactors. 
The results of this work are intended to
aid and guide policy makers and key stake holders on how to best establish a 
fuel cycle to support the deployment of \gls{HALEU}-fueled reactors. 
Within this primary goal, there are three specific objectives:
\vspace{0.2cm} 
\noindent
\begin{enumerate}
\item Quantify potential material requirements for the transition 
from \glspl{LWR} to advanced reactors in open and closed 
fuel cycles.

\item Understand the impacts of fuel cycle parameters on the material 
requirements and design optimized transition scenarios.

\item Identify potential limitations in using downblended \gls{HEU} 
on reactor performance.

\end{enumerate}
The goals of this work will be completed through the following steps:
\vspace{0.2cm} 
\noindent
\begin{enumerate}
\item \textbf{Model the transition from \glspl{LWR} to advanced reactors.} 
Transition scenarios to multiple fuel cycles with \gls{HALEU}-fueled 
reactors will 
be modeled using the \gls{NFC} simulator \Cyclus \cite{huff_fundamental_2016}. 
Multiple transitions will be modeled, with each one varying based on the type
of  
advanced reactors, energy demand, and fuel cycle option (open or closed). This 
step will quantify and compare the material requirements of each transition 
scenario to understand how the material requirements compare to what is available 
through the current US fuel cycle and published estimates of potential 
\gls{HALEU} needs. 

\item \textbf{Perform sensitivity analysis and optimize select transition scenarios.}
Sensitivity analysis will be performed on a subset of the transition scenarios
modeled by coupling \Cyclus with Dakota \cite{adams_dakota_2019}. This step 
will identify the most impactful model parameters on 
the material requirements of the fuel cycle. These results will then be 
used to determine optimized transition scenarios to minimize specific 
material requirements. 

\item \textbf{Neutronics analysis of \gls{HALEU} created from impure \gls{HEU}.}
The neutronics of the \gls{HALEU}-fueled reactors in the transition scenarios 
will be modeled to identify potential effects of using \gls{HALEU} produced 
from downblended \gls{HEU} with known impurities. This step will investigate 
potential limitations in using this \gls{HALEU} production method and 
how different \gls{HALEU} production methods affect reactor operation and 
key safety parameters, such as neutron multiplication, delayed neutron 
fractions, and reactivity coefficients.
\end{enumerate}


The following chapters present the work done to complete each of 
these steps and accomplish each objective. Chapter 2 
provides some 
background information and literature review on the nuclear fuel cycle,
nuclear fuel cycle 
modeling efforts, and the use of \gls{HALEU} in reactors.
Chapter 3 discusses the methodology for modeling the transition to advanced 
reactors in a once-through and closed fuel cycles. This chapter includes 
the scenarios modeled, information on the reactors included in the 
transitions, and the flow of 
material in each scenario. The scenarios modeled vary based on the 
advanced reactors deployed, the energy demand of the scenario, and the 
type of fuel cycle (open or closed). This chapter also includes discussion 
of the development of OpenMCyclus, a code that couples \Cyclus with OpenMC 
to provide real-time depletion of fuel during a fuel cycle simulation.

Chapter 4 discusses the results of the 
once-through fuel cycle transition scenarios. Chapter 5 discusses the 
results of the closed fuel cycle transitions. For each of the transition 
scenarios modeled, we compared them based on the number of reactors 
deployed, and various material requirements. Material requirements 
considered include the total mass of enriched uranium, mass of \gls{HALEU}, 
feed mass to enrich uranium, the \gls{SWU} capacity required to enrich 
uranium, and the mass of \gls{SNF} discharged from the reactors. 

Chapter 6 provides results 
and analysis of the sensitivity analysis performed for select once-through 
and closed fuel cycle transition scenarios. Model parameters considered 
for the sensitivity analysis include the transition start time, 
the percent of \glspl{LWR} that receive license extensions, the build share 
of each advanced reactor, and the discharge fuel burnup of the 
\gls{HALEU}-fueled advanced reactors. The analysis considers each model 
parameters effect and their combined effects from varying multiple 
parameters on the same material requirements considered in Chapters 4 and 5.


Chapter 7 provides 
results and discussion of the optimization of select once-through and 
closed fuel cycle transitions. The fuel cycles are optimized for minimizing 
the \gls{SWU} capacity required to produce \gls{HALEU}, minimize the 
mass of \gls{SNF} disposed of, and both in a multi-objective problem. This 
portion of the work is done by coupling \Cyclus with Dakota 
\cite{adams_dakota_2021}, and the discussion includes analysis of the results 
and the performance of this methodology for optimizing \glspl{NFC}. 

Chapter 8 provides the methodology and 
results of evaluating the performance of downblended \gls{HEU} in 
the \gls{HALEU}-fueled advanced reactors. Metrics considered for the 
reactor performance include the \keff, \betaEff, neutron flux, 
and multiple reactivity temperature feedback coefficients. 
Finally, Chapter 9 provides concluding remarks. 