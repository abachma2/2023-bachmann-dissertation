\section{Summary}
The purpose of this work is to investigate the impacts of 
deploying \gls{HALEU}-fueled advanced reactors in the United States. 
The literature 
review summarized different \gls{NFC} options, tools used to perform 
\gls{NFC} analysis and sensitivity analysis, 
the current understanding of the need for \gls{HALEU}, and known 
effects of using \gls{HALEU} in reactors on the nuclear fuel cycle 
and reactor performance. The methodology chapter 
described how models of advanced reactors and once-through fuel cycle 
simulations were constructed, leading to the results presented in the 
fourth chapter. 

The work presented models the transition from the current fleet of 
\glspl{LWR} in the US to various combinations of advanced reactors, 
specifically the X-energy Xe-100, \gls{USNC} \gls{MMR}, and the 
NuScale VOYGR, assuming either a no growth or 1\% annual growth in 
energy demand. The results from these scenarios include the material 
requirements of the transitions, specifically the number of reactors 
deployed to meet a prescribed energy demand, the 
uranium resources, the \gls{SWU} 
capacity, and the waste produced. The number of reactors required 
in each transition scenario scales based on the power output of the 
reactors and the energy demand of the scenario. The uranium resources, 
\gls{SWU} capacity and waste discharged are 
affected more by the reactors deployed than the energy demand. Deploying 
the 
NuScale VOYGR (an advanced reactor that does not require \gls{HALEU}) 
in tandem with the X-energy Xe-100 or \gls{USNC} \gls{MMR} (reactors 
that require \gls{HALEU}) reduces the need for resources related to 
\gls{HALEU}, but does not necessarily decrease the total amount of 
enriched uranium required. 
The mass of enriched uranium required and the mass of waste discharged 
increase when deploying the VOYGR alongside the other reactors, but the 
feed uranium mass and \gls{SWU} capacity are comparable to when the VOYGR 
is not deployed. These results 
highlight how the mass for a core loading and the 
uranium assay of the fuel for each reactor type affect the material 
requirements. The 
quantification and understanding of these material needs aid in 
establishing fuel cycle facilities and resources to meet the needs of 
the transition. 

\subsection{Corrected deployment methodology}
To correct the flaw in the advanced reactor deployment methodology, 
the deployment scheme 
will be altered to match energy produced to the energy demand, instead 
of matching installed capacity to energy demand. Making this change 
involves updates to the capacity factor of each reactor type to 
accurately capture the energy each reactor type will produce. 

The \gls{LWR} fleet is currently modeled with a 94.7\% capacity factor, 
which is higher than historic data of actual capacity factors 
\cite{us_eia_monthly_2022}, as discussed in Section \ref{sec:scenario1}.
In all updated and future work 
I will use a capacity factor of 92.66\% for the \glspl{LWR}, which 
is an average of the reported capacity factors for the \gls{LWR} fleet 
over the last 5 years \cite{noauthor_electric_2022}. The NuScale VOYGR 
is also currently modeled with 
a 94.7\% capacity factor, which will be updated to 95\%, based on 
NuScale's projected capacity factor for the reactor 
\cite{nuscale_technology_nodate}. The Xe-100 and \gls{MMR} are currently 
modeled as a 100\% capacity factor, because they are each modeled with a 
zero-month refueling duration. The Xe-100 will be modified to a 95\% 
capacity factor, based on estimates from X-energy 
\cite{xenergy_reactor_nodate}. The \gls{MMR} will continue to be modeled 
with a 100\% capacity factor because the difference between 
5 MWe and 4.75 MWe (power output when using a 95\% capacity factor) is 
not expected to have any meaningful effect on the results of this work. 

To account for each of these capacity factors, two changes need to be 
made 
to the definition of each reactor type. First, the power output of each 
reactor type is multiplied by the appropriate capacity factor. This 
changes the power output of the Xe-100 to 76 MWe and the power output of 
the VOYGR to 73 MWe. The power output of each of the \glspl{LWR} is 
decreased to 92.66\% of the power listed in the \gls{PRIS} database. 
Second, the VOYGR and \glspl{LWR} are altered to have a refueling 
duration of zero months and their operating cycle is extended by one 
month. This alteration ensures that only the specified capacity factor 
is applied
to each of these reactors, and limitations in the modeling capabilities 
do not affect the capacity factor or introduce artificial changes to the 
reactor capacity factor. By extending the operating 
cycle by one month, the reactors still receive fuel on the same schedule 
as when the refueling outages were explicitly modeled. 

These changes to the modeling methodology are expected to affect each of 
the results of this work. First, the energy produced by the \gls{LWR} 
fleet in 
2025 is expected to change, because they are now modeled to have a 
smaller maximum power output. 
This change in the results will affect the energy demand of each of the 
transition scenarios, thus affecting the number of advanced reactors 
deployed and the material requirements of the advanced reactors. This 
new methodology does not affect the amount of fuel and resources required 
by each type of reactor or when they require them. 


\section{Proposed Work}
The preliminary work accomplishes part of the first objective of this work: 
quantify potential material requirements for the transition from 
\glspl{LWR}
to advanced reactors in open and closed fuel cycles. The results identify 
potential material needs for transitioning to various combinations of 
advanced reactors assuming different energy demands. However, this work 
is very limited in scope, and does not investigate the material requirements 
of these transitions with a closed fuel cycle or how model parameters 
other 
than the reactor types and energy demand affect the results. Additionally, 
this work only considers the enrichment of natural uranium to produce 
\gls{HALEU}. 

The proposed work expands upon the once-through transition analysis in 
evaluating the effects of deploying \gls{HALEU}-fueled reactors in the 
United States. 
To investigate how changing the fuel cycle to a closed fuel affects the 
material requirements, the transition to advanced reactors in a closed fuel 
cycle will be modeled. This step will complete the first 
objective of this work. These scenarios and the once-through scenarios 
investigate how the reactors, energy demand, and fuel cycle type affect 
the material requirements. However, previous fuel cycle analysis has 
shown that other model parameters, such as the transition start time 
and lifetime of existing reactors, will also affect the material 
requirements. Therefore, 
the next stage of proposed work will perform sensitivity analysis on 
select once-through and recycle fuel cycles modeled. The results of the 
sensitivity 
analysis will then used to develop optimized transition scenarios based 
on minimizing specific material requirements. This section of work will 
complete the second objective of this work. Finally, downblending 
\gls{HEU} is a viable 
method to produce \gls{HALEU}, but impurities in the \gls{HALEU} may limit 
the amount of downblended \gls{HEU} that can be present in a reactor core 
at once. To investigate potential limitations on using downblended 
\gls{HEU} in the advanced reactors, this work will perform depletion 
calculations and neutronics analysis of reactor performance using 
\gls{HALEU} produced by 
downblending \gls{HEU}. This step will complete the third and final
objective of this work. 

\subsection{Recycle Fuel Cycles}
Transition scenarios with a closed fuel cycle will be created
to investigate how this type of \gls{NFC} affects the material requirements 
and how the recycling scheme impacts them. 
The scenarios investigated in this analysis vary by the 
energy demand of the scenario and the recycling scheme (Table 
\ref{tab:scenarios_recycle}). Energy demands vary 
between a no growth and a 1\% growth model, the same as the variation in this 
parameter in the once-through fuel cycle models. Variations in the recycling 
scheme include either a limited or a continuous recycle. Limited recycle 
scenarios assume that \gls{SNF} is recycled once and disposed of after a 
second pass through the reactor. Continuous recycling assumes that all 
\gls{SNF} is recycled an unlimited number of times until all fissile 
material has been used. Both recycling schemes were considered in the 
\acrfull{ES} \cite{wigeland_nuclear_2014}, which provides the basis for why 
both schemes are considered in this work.  
All of the scenarios with recycling will model the transition from the 
current fleet of \glspl{LWR} to the X-energy Xe-100, \gls{USNC} \gls{MMR}, 
and the NuScale VOYGR, the same as in scenarios 7 and 13. This combination 
of reactors is considered for this step of the work, and not all of the 
combinations considered for the once-through scenarios, to limit the scope 
of the work while providing some amount of comparison between the 
once-through and recycle transition scenarios. 

\begin{table}[ht]
    \centering
    \caption{Summary of the recycle fuel cycle transition scenarios.}
    \label{tab:scenarios_recycle}
    \begin{tabular}{l l l}
            \hline
            Scenario number & Energy growth model & Recycle scheme\\\hline
            14 & No growth & Limited \\
            15 & No growth & Continuous \\
            16 & 1\% growth & Limited \\
            17 & 1\% growth & Continuous \\
            \hline
    \end{tabular}
\end{table}

Figure \ref{fig:limited_recycle_flow} shows the fuel cycle and material flows 
for the scenarios with limited recycling (Scenarios 14 and 16). This fuel 
cycle assumes that \gls{SNF}
from the \glspl{LWR} and advanced reactors can undergo separation and be 
recycled. A separations facility is defined to be deployed 5 years before 
the transition 
begins (i.e. in 2020), as this strategy is commonly used for modeling 
transitions with recycling \cite{passerini_systematic_2014,richards_application_2021}
to ensure that enough fuel can be separated and 
processed in time for use in advanced reactors. Although this is a 
non-physical time to begin recycling in the US, using this timeline ensures 
that there will be enough reprocessed fuel to fuel all of the advanced 
reactors. The deployment time of the separation facility will be adjusted 
to determine if a more physical deployment timeline is possible. Additionally, 
by using this timeline, specifically ensuring that advanced reactors 
are deployed at the same time as when they are deployed in the once-through 
transition to provide an even comparison between these fuel cycle options. 

\begin{figure}
    \centering
    \begin{tikzpicture}[node distance=1.5cm]
        \node (mine) [facility, text width=1cm] {Uranium Mine};
        \node (enrichment) [facility, below of=mine]{Enrichment};
        \node (reactor) [facility, below of=enrichment]{LWR};
        \node (mmr) [transition, right of=reactor, xshift=1.5cm]{MMR};
        \node (xe100) [transition, right of=mmr, xshift=1.5cm]{Xe-100};
        \node (voygr) [transition, right of=xe100, xshift=1.5cm]{VOYGR};
        \node (wetstorage) [facility, below of=reactor, text width=1cm]{Cooling Pool};
        \node (drystorage) [facility, below of=wetstorage, text width=1.5cm]{Dry Storage};
        \node (cooling) [transition, below of=mmr, xshift=1.5cm, text width=1cm]{Cooling Pool};
        \node (sinkhlw) [facility, below of=drystorage, xshift=2.5cm, yshift=-1cm]{Repository};
        \node (sinkllw) [facility, left of=enrichment, xshift=-1.5cm, text width=1cm]{Tails Sink};
        \node (separation) [transition, below of=cooling, yshift=-1cm]{Separations};
        \node (mox_fab) [transition, below of=voygr,xshift=2cm, text width=1cm]{MOX Fuel Fab};
        \node (mox_cooling) [transition, below of=xe100, xshift=1.5cm, yshift=-1.5cm]{MOX Cooling};
        
        \draw [arrow] (mine) -- node[anchor=west]{Natural uranium} (enrichment);
        \draw [arrow] (enrichment) -- node[anchor=east]{Fresh UOX}(reactor);
        \draw [arrow] (enrichment) -- node[anchor=south]{Tails}(sinkllw);
        \draw [arrow] (enrichment) -| (mmr);
        \draw [arrow] (enrichment) -| node[anchor=south]{Fresh UOX}(xe100);
        \draw [arrow] (enrichment) -| (voygr);
        \draw [arrow] (reactor) -- node[anchor=east]{Spent UOX}(wetstorage);
        \draw [arrow] (wetstorage) -- node[anchor=east]{Cool Spent UOX}(drystorage);
        \draw [arrow] (drystorage) |- node[anchor=east]{Casked Spent UOX}(sinkhlw);
        \draw [arrow] (mmr) |- node[anchor=east]{Spent UOX}(cooling);
        \draw [arrow] (xe100) |- (cooling);
        \draw [arrow] (voygr) |- node[anchor=south]{Spent UOX}(cooling);
        \draw [arrow] (xe100) -| (mox_cooling);
        \draw [arrow] (voygr) -| node[anchor=south, text width=1cm, pos=0.25]{Spent MOX}(mox_cooling);
        \draw [arrow] (cooling) -- node[anchor=west, text width=1.5cm]{Cooled Spent UOX}(separation);
        \draw [arrow] (separation) -| node[anchor=south,text width=1.5cm, pos=0.4]{Separated fissile material}(mox_fab);
        \draw [arrow] (separation) -| node[anchor=south, text width=1.5cm]{Fission Products}(sinkhlw);
        \draw [arrow] (mine) -| node[anchor=south, pos=0.4]{Natural uranium} (mox_fab);
        \draw [arrow] (mox_cooling) |- node[anchor=south]{Cool Spent MOX}(sinkhlw);
        \draw [arrow] (mox_fab.east) - ++(5mm,0) |- node[anchor=south, pos=0.5, text width=1.5cm]{Fresh MOX}(voygr.east);
        %\draw [arrow] (mox_fab.east) -| ++(5mm,0) -| (xe100.north);
        \end{tikzpicture}
    \caption{Fuel cycle facilities and material flow between facilities for modeling the transition 
    to advanced reactors with a limited recycle fuel cycle. The separations facility
    is deployed in 2020, five years before the transition. Before 2020 a once-through 
    fuel cycle is used with the facilities in blue.}
    \label{fig:limited_recycle_flow}
\end{figure}

Figure \ref{fig:continuous_recycle_flow} shows the facilities and material 
flow for the continuous recycle fuel cycle. Continuous recycle 
requires a fast reactor, and all of the reactors considered in this 
work have a thermal neutron energy spectrum. Therefore, a fast reactor 
facility 
must be added into the transition. To accomplish this, two different 
transitions are included in these scenarios: the transition from 
\glspl{LWR} to the advanced reactors and the transition from the 
advanced reactors to the fast reactor. The first transition will begin 
in 2025, and the second transition will begin in 2090, and the deployment 
rate of each reactor type will be based on the energy demand of the 
scenario. Because of the lifetime of the advanced reactors, these scenarios 
will be extended out to 2155 to assume that advanced reactors would not 
be prematurely decommissioned to facilitate this transition. These 
scenarios will consist of a longer time period than the once-through 
and limited recycle scenarios, which leads to uneven comparisons. Therefore, 
the results of the continuous recycle scenarios will focus on the 
material requirements after 2090, to provide a consistent 65 year 
time period for comparison. 
The fast reactor will be modeled 
based on the fast reactor technology used in the \gls{ES} 
\cite{wigeland_nuclear_2014} to provide a generic representation of 
a fast reactor and not any specific commercial design of fast reactor 
technology. 

\begin{figure}
    \centering
    \begin{tikzpicture}[node distance=1.5cm]
        \node (mine) [facility] {Uranium Mine};
        \node (enrichment) [facility, below of=mine]{Enrichment};
        \node (reactor) [facility, below of=enrichment]{LWR};
        \node (adv_reactor) [transition, right of=reactor, xshift=3cm]{Advanced Reactor};
        \node (wetstorage) [facility, below of=reactor]{Wet Storage};
        \node (drystorage) [facility, below of=wetstorage]{Dry Storage};
        \node (cooling) [transition, below of=adv_reactor]{Cooling Pool};
        \node (sinkhlw) [facility, below of=drystorage, xshift=2.5cm]{Repository};
        \node (sinkllw) [facility, left of=enrichment, xshift=-3cm]{Tails sink};
        \node (separation) [transition, below of=cooling]{Separations};
        \node (fuelfab) [transition, below of=adv_reactor,xshift=3cm]{Fuel Fab};
        \node (sfr) [transition, right of=adv_reactor, xshift=3.5cm]{Fast Reactor};

        \draw [arrow] (mine) -- node[anchor=east]{Natural U} (enrichment);
        \draw [arrow] (enrichment) -- node[anchor=east]{Enriched U}(reactor);
        \draw [arrow] (enrichment) -- node[anchor=south]{Tails}(sinkllw);
        \draw [arrow] (enrichment) -| node[anchor=west]{Fresh Fuel}(adv_reactor);
        \draw [arrow] (reactor) -- node[anchor=east]{Spent UOX}(wetstorage);
        \draw [arrow] (wetstorage) -- node[anchor=east]{Cool Spent UOX}(drystorage);
        \draw [arrow] (drystorage) |- node[anchor=east]{Casked Spent UOX}(sinkhlw);
        \draw [arrow] (adv_reactor) -- node[anchor=west]{Spent Fuel}(cooling);
        \draw [arrow] (cooling) -- node[anchor=west]{Cooled Spent Fuel}(separation);
        \draw [arrow] (separation) -| node[anchor=north, text width=1.5cm]{Separated fissile material}(fuelfab);
        \draw [arrow] (fuelfab) |- node[anchor=south, text width=1.5cm]{Reprocessed fuel}(sfr);
        \draw [arrow] (fuelfab) |- (adv_reactor);
        \draw [arrow] (separation) |- node[pos= 0.3, anchor=west, text width = 1.5cm]{Fission products}(sinkhlw);
        \draw [arrow] (wetstorage) -- node[sloped, anchor=south]{Cool Spent UOX}(separation);
        \draw [arrow] (sfr) |- node[anchor=west]{Waste}(sinkhlw);
        \draw [arrow] (mine) -| node[anchor=south]{Natural uranium}(sfr);

        \end{tikzpicture}
    \caption{Fuel cycle facilities and material flow between facilities for modeling the transition 
    to advanced reactors with a continuous recycle fuel cycle. Before 2020 a once-through 
    fuel cycle is used with the facilities in blue. Facilities in red are deployed 
    in 2025, and represent all of the facilities and material flows in red in 
    Figure \ref{fig:limited_recycle_flow}.}
    \label{fig:continuous_recycle_flow}
\end{figure}

\subsection{Sensitivity Analysis and Optimization}
The sensitivity analysis identifies how the variation of model input
parameters affect select performance 
metrics of the fuel cycle. Additionally, it reveals the combined effect 
of varying multiple parameters on 
an output metric. This information is then used to design 
an optimized transition scenario by identifying which input parameters 
will affect the results the most, and how these parameters should be 
changed to obtain desired results (e.g., minimizing \gls{HALEU} requirements
of a transition). The analysis will be performed on Scenario 7 and Scenario 
14 to provide insight into how the input parameters affect the output 
metrics for both a once-through and recycle fuel cycle. 

To perform the sensitivity analysis and 
transition optimization \Cyclus will be coupled to Dakota \cite{adams_dakota_2019},
and Dakota will be the driver for the analysis. This coupling will be based on 
the scripts in the \texttt{dcwrapper} GitHub repository \cite{chee_arfc/dcwrapper_2019}.
Three different types 
of sensitivity analysis will be performed: \acrfull{OAT}, synergistic, 
and global. \gls{OAT} analysis will vary a single input parameter to 
investigate the effect of each parameter individually. Synergistic 
analysis will vary two input parameters at once to investigate how the 
interaction of the two parameters affects the results. Finally, the global 
analysis will vary more than two input parameters to provide a holistic 
view of how multiple parameters interact and affect the output metrics. 

The input parameters to be varied include the transition 
start time, the build share of each type of advanced reactor, and 
the \gls{LWR} lifetime. The transition start time will range from January 
2025 to January 2040 in three-month intervals, but the same energy demand 
will be specified for all perturbations (89.45 GWe-yr). The build share 
variable will be repeated three times (for varying the build share of each 
type of advanced reactor) with build share percentages ranging from 0-50\% 
in increments of 5\%. The remainder of the energy capacity will be met by 
deploying the other advanced reactors in a similar fashion to how they 
are deployed in the once-through transition scenarios. 

Finally, the \gls{LWR} lifetimes will be varied based 
on the percent of the fleet that operate for 60 and 80 years. This 
variable will vary between 0-50\% of the \gls{LWR} fleet operating for 80 
years, while the other \glspl{LWR} operate for 60 years. The 
\glspl{LWR} do not all start operation at the same time, so the 
selection of the \glspl{LWR} that operate for 80 years will impact the results, 
even if the number is the same. Therefore, reactors will be selected in 
descending order of power output, reflecting the greater likelihood of 
larger units receiving a license extension. Previous sensitivity analysis of 
fuel cycle transitions considered the impact of the transition start time 
and the \gls{LWR} lifetimes \cite{chee_sensitivity_2019,feng_sensitivity_2020},
which forms the basis for why these input parameters were selected for this 
work.

The output metrics of 
interest for this analysis include the amount of waste generated that 
must be sent to a repository, the mass of enriched uranium, the mass of 
\gls{HALEU}, and 
the amount of \gls{SWU} capacity required to produce all enriched uranium, the 
\gls{SWU} capacity required to produce \gls{HALEU}, and the feed uranium 
required to produce \gls{HALEU}. Each metric will be evaluated based on the 
cumulative sum required, starting at the transition start time. Previous 
sensitivity analysis of fuel cycle transitions considered the waste 
discharged, \gls{SWU} 
capacity required, and natural uranium requirements
\cite{richards_application_2021,feng_sensitivity_2020} 

The results of the \gls{OAT} analysis will be reported as figures 
to describe general trends and statistics (maximum, minimum, and 
average) to quantify the effects of the input parameter. 
The results 
of the synergistic analysis will be reported using surface plots show 
the relationship 
between the inputs and the output metric, similar to results presented in 
\cite{chee_sensitivity_2019} and \cite{passerini_systematic_2014}.
The results of the global analysis will be reported using Sobol' 
indices, which have been used in \cite{richards_application_2021}. 

The scenarios will then be optimized based on specific goals, to be 
identified after completion of the sensitivity analysis. 
Potential 
optimization problems include minimizing the \gls{HALEU} mass required 
or minimizing \gls{SNF} produced in Scenarios 7 and 14. Two 
single-objective and one multi-objective optimization problems will be 
constructed for this work.  

\subsection{Neutronics Analysis}
Recovery and downblending of \gls{HEU} to produce \gls{HALEU} means that 
the fuel is likely to contain impurities present in the original
\gls{HEU}. Known impurities in potential \gls{HEU}
supplies to create \gls{HALEU} include $^{232}$U and $^{236}$U
\cite{vaden_isotopic_2018,nelson_foreign_2010},  
which are parasitic neutron absorbers and have the potential to affect 
reactor physics and reactor operation. To investigate the magnitude of this 
effect, this step will model the use of 
impure \gls{HALEU} fuel in a reactor and compare it to the use of pure 
\gls{HALEU} (only $^{235}$U and $^{238}$U)
to understand how the impurities affect the performance of the reactor. 

Full core models of the Xe-100 and \gls{MMR} will be created in Serpent 
\cite{leppanen_serpent_2014} to model the neutronics and depletion of 
each reactor 
design. The fresh fuel composition for each core will be varied to provide 
a comparison of pure and impure fuel.
Material compositions of impure fuel will be modeled based on the 
known compositions of \gls{HALEU} from the \gls{EBR} stockpile 
\cite{vaden_isotopic_2018} and the Y-12 stockpile \cite{nelson_foreign_2010}.
Each core will be modeled three times: once modeling pure \gls{HALEU}, 
once modeling impure \gls{HALEU} from the \gls{EBR} stockpile, and once 
modeling impure \gls{HALEU} from the Y-12 stockpile.
By modeling the core composition to be entirely composed of impure fuel 
this work reports the most extreme effect that these impurities can 
have on reactor performance. Understanding the limitations of using only 
impure \gls{HALEU} informs on if downblended \gls{HEU} can be used in 
these reactors exclusively for initial reactor core loadings, or if 
infrastructure for enriching natural uranium must be established to 
produce \gls{HALEU} for initial core loadings. 

Results of each simulation to be compared include the neutron flux at 
\gls{BOL}, mid-cycle, and \gls{EOL}, $k_{eff}$ as a function of burnup, 
fuel temperature reactivity coefficient, and 
$\beta_{eff}$ as a function of burnup. Each parameter provides a 
measurement of the performance of 
the reactor, such as the materials degradation rate, amount of burnable 
poisons required, control rod worth, and the cycle time. Investigating 
each of these results helps to determine if the impurities potentially 
present in \gls{HALEU} will prevent any of the design criteria of the 
reactors from being met. 