This chapter reports the results for the recycle fuel cycle 
scenarios (Scenarios 14-19) described 
in Section \ref{sec:recycle-methods}. The primary results considered 
for these fuel cycle transitions are the uranium resources needed, 
the \gls{SWU} capacity requried, the separated plutonium masses, 
and the mass of material disposed of. This chapter does not focus 
as much on the number of reactors or the energy supplied by 
the reactors because most of the scenarios use the same 
deployment scheme as Scenarios 7 or 14 (depending on 
the energy demand curve of the scenario). For the two 
scenarios that deploy the \gls{SFR} instead of the other 
advanced reactors (Scenarios 16 and 19), the maximum number 
of \glspl{SFR} deployed in each scenario is 312 and 595, 
respectively. 
These scenarios require far fewer reactors than the other scenarios 
because the \gls{SFR} has a larger power output than the other 
advanced reactors (311 MWe compared with 80 MWe for the Xe-100).

\section{Uranium resources}
The uranium resources described here are divided into two primary 
components: the heavy metal mass and the feed uranium required 
to produce enriched uranium. The heavy metal mass is further 
broken into two parts: the mass of enriched uranium and the 
mass of plutonium-based fuel. 

\subsection{No growth scenarios}

\subsection{1\% growth scenarios}

\section{SWU capacity}

\subsection{No growth scenarios}

\subsection{1\% growth scenarios}

\section{Separated plutonium}

\subsection{No growth scenarios}

\subsection{1\% growth scenarios}

\section{Spent nuclear fuel and High Level Waste}
This section provides the results of the \gls{SNF} and \gls{HLW} sent 
to disposal in the closed fuel cycles. The once-through scenario 
results only defined the \gls{SNF} sent for disposal because all 
of the materials needing disposal in those scenarios are \gls{SNF}. 
With the addition of the separation step, the \gls{SNF} discharged 
from reactors gets separated into two different material streams 
that are \gls{HLW}. This \gls{HLW} also needs to be disposed of, 
which is why it is considered as part of these results. 

\subsection{No growth scenarios}

\subsection{1\% growth scenarios}