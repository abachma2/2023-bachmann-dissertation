\begin{frame}
    \frametitle{Downblending HEU is a potential source of HALEU}
    To meet the fourth objective, I modeled the neutronics of 
    different HALEU compositions in the Xe-100 and MMR.
    \begin{itemize}
        \item Consider pure HALEU ($^{235}$U and $^{238}$U only)
              and HALEU from downblended \gls{EBR} 
              \cite{vaden_isotopic_2018} and Y-12 
              \cite{nelson_foreign_2010}  \gls{HEU} inventories
        \item Create models in Serpent \cite{leppanen_serpent_2013} 
              for these two reactors
        \item Compare the performance of the fuels with respect to:
        \begin{itemize}
            \item \keff
            \item \betaEff
            \item Energy- and spatially-dependent flux
            \item Fuel, coolant, moderator, and total reactivity
                  temperature feedback coefficients
        \end{itemize}
    \end{itemize}

\end{frame}

\begin{frame}
    \frametitle{\keff and \betaEff of Xe-100}
    \begin{columns}
        \column[t]{5cm}
    \begin{itemize}
        \item Impurities increase \keff, larger $\eta$ value 
        ($\frac{\nu\sigma_{f}}{\sigma_a}$) of the fuel 
        \vspace{2.5cm}
        \item<2-> Impurities decrease \betaEff because non-$^{235}$U 
              isotopes have a smaller \betaEff than $^{235}$U
    \end{itemize}
        \column[t]{5cm}
        \begin{table}[ht]
            \centering 
            \caption{\keff values for the Xe-100-like reactor model using
            each fuel type.}
            \label{tab:xe100_keff}
            \begin{tabular}{c c}
                    \hline
                    Fuel type & \keff \\
                    \hline 
                    Pure & 1.06663 $\pm$ 0.00016\\
                    \gls{EBR} & 1.08086 $\pm$ 0.00016\\
                    Y-12 & 1.08016 $\pm$ 0.00014\\
                    \hline                
            \end{tabular}
        \end{table}

        \pause
        \begin{table}[ht]
            \centering 
            \caption{\betaEff value for the Xe-100-like reactor 
            mode using each fuel type.}
            \label{tab:betaeff_xe100}
            \begin{tabular}{c c}
                    \hline
                    Fuel type & \betaEff \\
                    \hline
                    Pure & 0.00617 $\pm$ 0.00003 \\
                    \gls{EBR} & 0.00604 $\pm$ 0.00003 \\
                    Y-12 & 0.00598 $\pm$ 0.00003 \\
                    \hline
            \end{tabular}
        \end{table}
    \end{columns}

\end{frame}

\begin{frame}
    \frametitle{Energy dependent neutron flux in Xe-100}
    \begin{figure}
        \centering 
        \includegraphics[scale=0.5]{xe100_mg_flux.pdf}
        \caption{Energy-dependent flux through the active region 
        of the Xe-100 core. The purple line is the delineation 
        between fast and thermal neutrons for this work.}
    \end{figure}
\end{frame}

\begin{frame}
    \frametitle{Spatitally-dependent neutron flux in Xe-100}
    \begin{figure}
        \centering 
        \begin{subfigure}{0.49\textwidth}
            \includegraphics[scale=0.35, trim=20 10 10 20,clip]{xe100_thermal_radial.pdf}
        \end{subfigure}
        \begin{subfigure}{0.49\textwidth}
            \includegraphics[scale=0.35, trim=10 10 10 20,clip]{xe100_fast_radial.pdf}
        \end{subfigure}
        \caption{Radial fluxes through Xe-100.}
        \label{fig:xe100-flux}
    \end{figure}
\end{frame}