% General Summary
The US \acrfull{NFC} is poised to change as a greater variety of 
nuclear reactor designs are licensed and deployed. One notable change 
between currently deployed nuclear reactors and many advanced 
reactor designs is the enrichment level required; many advanced 
reactors require fuel at a higher enrichment level (5-20\% $^{235}$U, 
referred to as \acrfull{HALEU}) than current \acrfull{LWR} technology 
(less than 5\% $^{235}$U). Using a higher enrichment level provides 
numerous benefits to reactor operation, such as longer cycle 
times and higher fuel burnup. However, there is currently no 
domestic way 
to produce commercial \gls{HALEU}, which has prompted investigations 
into how to develop supply chains for \gls{HALEU}
\cite{regalbuto_addressing_2020,dixon_estimated_2022}. Potential 
methods to produce \gls{HALEU} include the enrichment of natural 
uranium and the downblending of \acrfull{HEU}, with each 
method having limitations. Enriching natural 
uranium is limited by the facilities available to perform 
the enrichment, and downblending \gls{HEU} is limited by the 
amount of material available for downblending. Using either (or 
both) of these methods will lead to changes in the US \gls{NFC}. 


The goal of this work was to investigate the impacts of deploying reactors 
fueled by \gls{HALEU} in the United States. Within this primary goal, 
there are three specific objectives:
\vspace{0.2cm} 
\noindent
\begin{enumerate}
\item Quantify potential material requirements for the transition 
from \glspl{LWR} to advanced reactors in open and closed 
fuel cycles.

\item Understand the impacts of fuel cycle parameters on the material 
requirements and design optimized transition scenarios.


\item Identify potential limitations in using downblended \gls{HEU} 
on reactor performance.
\end{enumerate}

\noindent The first objective was met by designing and modeling various 
transitions 
to advanced reactors, considering a once-through and a closed 
fuel cycle (Chapters \ref{ch:fc_methods} - \ref{ch:recycle_results}). 
The second objective was met by performing sensitivity 
analysis and optimization to one of the once through fuel cycles
(Chapters \ref{ch:sa} and \ref{ch:optimization}). 
Finally, the third objective was met by modeling the two different 
\gls{HALEU}-fueled advanced reactors to investigate the impact of 
different uranium isotopic 
compositions of the \gls{HALEU} fuel on reactor performance (Chapter 
\ref{ch:neutronics}). 

Chapter 3 provided an overview of the methodology used for the fuel 
cycle analysis of this work. The methodology includes the 
advanced reactors. advanced reactor design specifications,
fuel cycles considered, and the 
advanced reactor deployment scheme. This chapter also 
describes OpenMCyclus, a new archetype that couples \Cyclus with 
OpenMC to dynamically perform depletion of fuel during a simulation. 
This archetype extends the depletion capabilities in 
\Cyclus by providing a coupling to an open-source code, and 
allowing the depletion to be reactor-agnostic. 
Benchmarking against the \Cycamore \texttt{Reactor} archetype, 
a recipe-based archetype, shows generally good agreement with 
regards to when materials are traded away. There 
are large differences in the separated plutonium inventory between 
these two archetypes because of differences in the 
depletion methodology: \Cycamore \texttt{Reactor} applies the 
same composition to spent fuel no matter how many cycles the 
fuel is irradiated while OpenMCyclus performs depletion on a 
single-cycle basis. This methodology difference leads to different 
amounts of separated plutonium available, which propagates 
to different amounts of \gls{MOX} fuel available for the 
reactors. The methodology in OpenMCyclus leads to less 
separated plutonium available than the \Cycamore 
\texttt{Reactor} methodology. 

% Once-through analysis
Chapter 4 presented the results of modeling the transition from 
\glspl{LWR} to different combinations of advanced reactors in 
a once-through fuel cycle. 
The results identify how 
the reactors deployed and their relative deployment numbers, obtained 
through the deployment scheme of this work, drives 
the enriched uranium mass required by a transition. The discharge
burnup of the reactors deployed drives the enriched uranium 
mass needed by each transition. The \gls{SWU} capacity and 
feed uranium requirements are driven by the amount of fuel required 
by the reactors and the enrichment level of the fuel. The results 
show how the increased fuel requirements of a reactor demand can 
be offset by the decrease in the enrichment level, leading to 
negligible changes in the \gls{SWU} capacity required. 

% Closed fuel cycle analysis
Chapter 5 presented the results of modeling the transition from 
\glspl{LWR} to advanced reactors in different closed fuel cycles. 
The results of this chapter show how the amount of material 
available for reprocessing impacts the masses of separated 
actinide material and the availability of plutonium-based 
fuels (\gls{MOX} or U/TRU fuel). The amount of plutonium-based 
fuels impacts the 
mass of enriched uranium, feed uranium, and \gls{SWU} 
capacity required, because the plutonium-based fuel displaces 
uranium-based fuel needed by the advanced reactors. Additionally,
the actinide elements 
separated from \gls{SNF} impacts the masses of \gls{HLW} and separated 
actinide material. The more actinide elements separated out from 
\gls{SNF}, the more separated material available and more 
plutonium-based fuel available. The separation of uranium from the 
\gls{SNF} is the primary driver of this impact, as 
\gls{SNF} is mostly uranium by mass. Finally, this chapter 
showed how using a closed fuel cycle can reduce \gls{HALEU} 
needs, but the lack of existing reprocessing infrastructure 
in the US means that using a closed fuel cycle will not help 
in reducing upfront \gls{HALEU} needs. 
These results, combined 
with the results of Chapter 4, show that potential demand for 
\gls{HALEU} and other resources to support a \gls{HALEU}-based 
fuel cycle are dependent on multiple variables and parameters 
of the fuel cycle. 

% Sensitivity analysis
Chapter 6 examined the effects of different input parameters for 
their effects on different output metrics for the fuel cycle transition from 
\glspl{LWR} to different advanced reactors by performing sensitivity 
analysis. We performed this analysis by coupling \Cyclus with Dakota 
and perturbing input parameters. We selected input parameters based 
on the results in Chapter 4, specifically how the 
relative number of each advanced reactor built and the burnup from 
each reactor drove many of the transition metrics. 
The \acrfull{OAT} analysis identified the 
trends from varying each input parameter independently, and how the deployment 
scheme modeled in this work impacts each of the results. The analysis also 
highlighted tradeoffs between different reactor designs based on 
the results of the metrics and the number of each advanced reactor 
built for a given input parameter. An example of the tradeoffs between 
advanced reactors is the Xe-100 
needing more \gls{HALEU} than the VOYGR, but a smaller total fuel mass. We identified 
the transition start time as not as impactful on 
the results as the other parameters. The synergistic 
analysis identified how some of the input parameters interact to affect the output 
metrics. Some of the combinations of input parameters (e.g., varying 
the \gls{LWR} lifetime and the VOYGR build share) had trends consistent 
across the input parameter space and consistent with the results of 
the \gls{OAT} analysis. Other combinations (e.g., varying the Xe-100 
burnup and the \gls{MMR} share) varied in their effect across the input parameter 
space because of interactions between the parameters. Finally, the global 
sensitivity analysis quantified the effect of different input parameters 
on the variance of the output metrics. All of the analysis 
identified the Xe-100 discharge burnup as the most impactful parameter
for almost every metric, regardless of which advanced reactor build share 
was varied. This result stems from the strong relationship between 
the Xe-100 burnup and the different metrics, as well as the methodology 
of the deployment scheme. Based on the deployment scheme, the number of 
Xe-100s deployed varies, no matter which advanced reactor is selected for 
build share variation. Therefore, the material needs of the 
Xe-100, which is closely tied to the discharge burnup, always 
affects the metrics if there is variation in the advanced reactor 
build shares. 

% Optimization
Chapter 7 demonstrated a methodology to optimize the once-through 
transitions and identified transitions that minimize different 
fuel cycle metrics. We used the same \Cyclus-Dakota coupling used 
for the sensitivity analysis, but changed the Dakota inputs 
to apply the the sinlge- and multi-objective 
genetic algorithms in Dakota to perform optimization. 
Results from this chapter show that minimizing the 
\gls{SWU} capacity to produce \gls{HALEU} requires maximizing the number of 
VOYGRs built and minimizing the mass of \gls{SNF} requires maximizing the number 
of Xe-100s built. Minimizing both of these fuel cycle metrics is a 
balance between deploying Xe-100s and VOYGRs. The results of  
all three optimization problems show agreement in maximizing the number 
of \glspl{LWR} operating for 80 years, maximizing the Xe-100 burnup,
and minimizing the \gls{MMR} build share. The \gls{MMR} burnup becomes
irrelevant if no \glspl{MMR} are built, so the results 
identified multiple values for this input parameter. The results 
from the optimization 
work was consistent with the results of the sensitivity analysis, but 
they did not fully meet expectations. The input parameters identified that 
were not subject to a linear constraint (i.e., not the advanced reactor 
build shares) matched expectations based on the trends of the sensitivity 
analysis and intuition. However, the genetic algorithms we used 
did not adhere well to the linear constraint of the advanced reactor 
build shares in all three of the problems. The parameters valued 
considered in each populations did not always adhere to the constraint, 
which led to the solution not meeting the constraint. Therefore, the 
results from the optimization work can not be taken at face value. 
The results are better used to identifying a relationship between the advanced 
reactor build shares than a specific set of parameters, because the 
optimization solutions aligned well with intuition and the sensitivity 
analysis results even if the linear constraint is not met. Additionally, 
we limited the number of evaluations for each optimization problem,
based on limited computational resources. Allowing the 
algorithm to run more evaluations may produce results that better 
meet expectations and the linear constraint. 

% Neutronics
Finally, Chapter 8 described the Xe-100-like and 
\gls{MMR}-like reactor models developed and 
used to analyze the performance of \gls{HALEU} produced from 
downblended \gls{HEU}. The downblended \gls{HEU} compositions 
were compared against the performance of \gls{HALEU} with only 
$^{235}$U and $^{238}$U, referred to as the ``pure'' fuel. 
Performance metrics considered include the 
\keff, \betaEff, energy- and spatially-dependent flux, and 
the fuel, coolant, moderator, and total reactivity temperature 
feedback coefficients. The results of the Xe-100-like 
reactor in an equilibrium state show that the impurities from 
the downblended 
\gls{HEU} fuel increases the \keff but decreases the \betaEff, 
compared with the results of the pure fuel. 
The energy-dependent flux shows a decrease in the thermal flux 
around 0.1 eV and an increase in the fast flux above 20 MeV 
when using the downblended \gls{HEU}. The spatially-dependent 
flux shows a decrease in the thermal (below 0.625 eV) and 
fast (above 0.625 eV) fluxes when using the impure fuels, 
across the radial and axial directions. 
The fluxes also showed an asymmetry 
in the axial fluxes for both energy groups. This asymmetry 
is a result of the distribution of pebbles at different 
burnup steps, but the decrease in the thermal axial 
flux is a result of the \gls{HALEU} composition.
 The reactivity temperature 
feedback coefficients from each fuel composition are within 
error of the coefficients from the pure fuel or are within 
the ranges presented by Mulder and Boyes 
\cite{mulder_neutronics_2020}.

The results of the \gls{MMR}-like model at \gls{BOL}, \gls{MOL}, 
and \gls{EOL} show that the different \gls{HALEU} compositions 
result in lower \keff values than the pure fuel at each burnup 
step. However, 
the reactor still has a \keff above 1 at the \gls{EOL} with all 
three \gls{HALEU} compositions. 
This result suggests that the change in \keff would not prevent this 
reactor from reaching its designed 20 year lifetime. The 
different \gls{HALEU} compositions resulted in \betaEff that 
are within error of the values from the pure fuel. The 
energy-dependent 
flux showed that the downblended \gls{HEU} mostly affects 
the fast (above 0.625 eV) flux, causing a small increase in 
the flux. The spatially-dependent flux showed some 
asymmetry in the axial direction from one of the downblended 
\gls{HEU} compositions at \gls{MOL}, but we observed no other large 
effects. The reactivity temperature feedback coefficients 
are mostly within error of the coefficients from the pure fuel. 
The coolant reactivity feedback coefficients from the downblended 
\gls{HEU} are outside error from those from the pure fuel, 
but these values have a much smaller magnitude than the other 
reactivity feedback coefficients and operate on a much 
longer time scale. Based on the performance metrics of 
both reactor designs with the different \gls{HALEU} 
compositions, the impurities present in the downblended 
\gls{HEU} may lead to small perturbations in performance, but do 
not lead to large changes in operation or prevent the 
reactors from operating in a safe state. 
 
\section{Limitations and Future Work}
The work performed here provides a foundation for continued analysis and 
exploration of the fuel cycle impacts of deploying \gls{HALEU}-fueled reactors. 
One limitation of this work is that is explores the impacts of 
deploying \gls{HALEU}-fueled reactors at a very macroscopic level. 
We investigated and compared the material requirements across the 
modeled time period as an aggregate, with an unlimited amount 
of resources, and without facility constraints.
Potential future work to address this limitation would model and 
explore the impacts of these reactors on a more microscopic level. 
This includes translating potential demands to facility capacities, 
numbers, and designs or comparing the material requirements across 
different time periods. For example, designing and comparing the 
centrifuge cascades required to produce the enriched uranium for 
each of the advanced reactors. This work could account for the 
different \gls{NRC} facility classifications based on the 
enrichment level of the handled material. This future work 
would also explore how to 
most efficiently develop and use different facilities to produce the 
enriched uranium. Another example would be comparing the \gls{HALEU} mass 
required in the first five years of deploying \gls{HALEU}-fueled 
reactors compared with demand after all \glspl{LWR} decommission. This 
analysis would provide more detailed information on material requirements 
during the cycle cycle transition and during equilibrium of the new 
fuel cycle. A third potential area of work could model the time requirements 
of different processes (e.g. time to fabricate a fuel assembly). 
More accurate modeling of the time requirements for each step would 
provide details of required lead times and how these requirements 
may impact material availability. 

Futhermore, this work also focused on the uranium and heavy metals for 
the fuel, but these are not the only fuel components or reactor 
cores. Another area of future work includes modeling non-fuel 
materials needed to support these transitions, such as the amount 
of reactor-grade graphite 
to be the moderators in the Xe-100 and \gls{MMR}. This analysis would 
provide insight into other potentially limiting supply chain 
requirements that are  
impactful on establishing these fuel cycles. Additionally, this work 
focused on waste materials that need a repository for disposal. 
But there are other waste forms, like \acrfull{LLW}, that also need 
disposal. The fuel cycle modeling 
methodology employed in this work provides a foundation 
for how to carry out both of these analyses. 

A third limitation of this work is the disregard for nonproliferation 
safeguards in the fuel cycle modeling. Nonproliferation safeguards 
are an important part of ensuring the peaceful uses of nuclear 
power, and are implemented across fuel cycle facilities. To 
address this limitation, one could develop 
a method to incorporate safeguards into the fuel cycle modeling
methodology demonstrated here. 
Examples of incorporating safeguards into fuel cycle modeling could 
be to restrict facility throughputs, minimize stockpiles of 
enriched uranium at a facility, minimize idle \gls{SWU} 
capacity, and limit the amount of material that can be transported 
at once. Each of these restrictions can affect the ability 
to produce enough fuel for the deployed reactors. Investigating 
the effects of safeguards on the fuel cycle provides more 
detailed metrics of how to develop and support a fleet of 
\gls{HALEU}-fueled reactors. 

A key component of this work is the development of OpenMCyclus 
to expand the dynamic depletion capabilities in \Cyclus. The 
continued development of OpenMCyclus would support this expansion.  
One area of future work is to benchmark this archetype against 
other fuel cycle simulators that also perform real-time 
depletion. This additional benchmark would help to identify 
how differences in material handling and depletion 
propagate together between the codes, and identify potential 
areas of improvement for this archetype. 
Additionally, OpenMCyclus focuses on the back-end of 
the fuel cycle, but the back-end of the fuel cycle connects 
to the front-end when modeling reprocessing. Therefore, an 
extension of this work would be the
exploration of how the transport capabilities in OpenMC can 
be used to determine fresh fuel compositions, such as 
how \gls{DYMOND} has a method to perform a criticality search 
\cite{richards_application_2021}, and subsequent development 
of a fuel fabrication archetype in this library. 
This capability would prevent the user from having to determine 
\gls{MOX} or U/TRU fuel compositions \textit{a priori} while 
still providing accurate compositions for fresh fuel in a 
reactor that meets key design criteria.  

The optimization methodology demonstrated in this work is 
limited by the optimization algorithms used. Specifically, the 
genetic algorithms struggled with adhering to the linear constraint 
on the advanced reactor build shares. Future work to address this 
limitation could apply other algorithms in Dakota that strictly 
adhere to linear constraints, like the derivative-free methods 
\cite{adams_dakota_2021}, or allow the genetic algorithms to 
run for more evaluations. Another avenue of future work could 
also change the input parameters to only consider one advanced 
reactor build share, which would remove the linear constraint. 
This avenue would allow the defined advanced reactor build share to 
vary across integer values in a given range, then deploy the other 
advanced reactors as needed to meet any 
remaining energy demand. Removing the linear constraint is 
expected to yield a solution that does not have unneeded installed 
capacity from advanced reactors. 

Future work can also expand upon the analysis of the downblended 
\gls{HEU} on the reactor performance. For example, modeling other 
\gls{HALEU}-fueled reactors that 
have announced an intent to use \gls{HALEU} created from the identified 
\gls{HEU} stockpiles, like the Oklo Aurora, would provide more 
practical analysis. Additionally, future work could expand 
the analysis to include other metrics, such as power peaking factors and 
power distributions in the core. 
Expanding the analysis to include these parameters would provide 
more details on how the impurities from the \gls{HEU} may 
affect reactor operation, such as the amount of various reactivity 
control mechanisms needed. One can also increase the 
model fidelity by varying temperatures across the core, 
adding burnable absorbers to the materials, or altering the 
geometry to better match with vendor information as it is made 
publicly available. These 
updates to the models would provide information 
about how the \gls{HALEU} composition would affect the 
reactor performance in a more realistic simulation. 
