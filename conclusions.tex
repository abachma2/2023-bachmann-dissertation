% General Summary
The US \acrfull{NFC} is poised to change as a greater variety of 
nuclear reactor designs are licensed and deployed. One notable change 
between currently deployed nuclear reactors and many advanced 
reactor designs is the enrichment level required; many advanced 
reactors require fuel at a higher enrichment level (5-20\% $^{235}$U, 
referred to as \acrfull{HALEU}) than current \acrfull{LWR} technology 
(less than 5\% $^{235}$U). Using a higher enrichment level provides 
numerous benefits to reactor operation, such as longer cycle 
times and higher fuel burnup. However, there is currently no way 
to produce commercial \gls{HALEU}, which has prompted investigations 
into how to develop supply chains for \gls{HALEU}
\cite{regalbuto_addressing_2020,dixon_estimated_2022}. Potential 
methods to produce \gls{HALEU} include the enrichment of natural 
uranium and the downblending of \acrfull{HEU}. Each of these 
methods have their limitations. Enriching natural 
uranium is limited by the facilities available to perform 
the enrichment, and downblending \gls{HEU} is limited by the 
amount of material available for downblending. Using either (or 
both) of these methods will cause changes in the US \gls{NFC}. 


The goal of this work was to investigate the impacts of deploying reactors 
fueled by \gls{HALEU} in the United States. Within this primary goal, 
there are three specific objectives:
\vspace{0.2cm} 
\noindent
\begin{enumerate}
\item Quantify potential material requirements for the transition 
from \glspl{LWR} to advanced reactors in open and closed 
fuel cycles.

\item Understand the impacts of fuel cycle parameters on the material 
requirements and design optimized transition scenarios.


\item Identify potential limitations in using downblended \gls{HEU} 
on reactor performance.
\end{enumerate}

\noindent The first objective was met by designing and modeling various 
transitions 
to advanced reactors, considering a once-through and a closed 
fuel cycle. The second objective was met by performing sensitivity 
analysis and optimization to one of the once through fuel cycles. 
Finally, the third objective was met by modeling the two different 
\gls{HALEU}-fueled advanced reactors to investigate the impact of 
different uranium isotopic 
compositions of the \gls{HALEU} fuel on reactor performance. 

Chapter 3 provided an overview of the methodology used for the fuel 
cycle analysis of this work. The methodology includes the 
advanced reactors and fuel cycles considered, as well as how the 
advanced reactor deployment scheme works. This chapter also 
describes OpenMCyclus, a new archetype that couples \Cyclus with 
OpenMC to perform real-time depletion of fuel during a simulation. 
This archetype extends the capabilities of real-time depletion in 
\Cyclus by providing a coupling to an open-source code, and by 
not making the depletion specific to a single reactor design. 
Benchmarking against the \Cycamore \texttt{Reactor} archetype, 
a recipe-based archetype, shows generally good agreement with 
regards to when materials are traded away. There 
are large differences in the separated plutonium inventory between 
these two archetypes because of differences in the 
depletion methodology: \Cycamore \texttt{Reactor} applies the 
same composition to spent fuel no matter how many cycles the 
fuel is irradiated while OpenMCyclus performs depletion on a 
single-cycle basis. This methodology difference leads to different 
amounts of separated plutonium available, which propagates 
to different amounts of \gls{MOX} fuel available for the 
reactors. 

% Once-through analysis
Chapter 4 presented the results of modeling the transition from 
\glspl{LWR} to different combinations of advanced reactors in 
a once-through fuel cycle. 
The results identify how 
the reactors deployed and their relative deployment numbers drives 
the enriched uranium mass required by a transition. This result stems 
from the different discharge burnups of the reactors, which drives the 
amount of fuel required by the reactors. The \gls{SWU} capacity and 
feed uranium requirements are driven by the amount of fuel required 
by the reactors and the enrichment level of the fuel. The results 
show how the increased fuel requirements of a reactor demand can 
be offset by the decrease in the enrichment level to lead to 
negligible changes in the \gls{SWU} capacity required. 

% Closed fuel cycle analysis
Chapter 5 presented the results of modeling the transition from 
\glspl{LWR} to advanced reactors in different closed fuel cycles. 

The results in Chapters 4 and 5 show that potential demand for 
\gls{HALEU} and other resources to support a \gls{HALEU}-based 
fuel cycle are dependent on multiple variables and parameters 
of the fuel cycle. 

% Sensitivity analysis
Chapter 6 examined the effects of different input parameters for 
their effects on different output metrics for the fuel cycle transition from 
\glspl{LWR} to different advanced reactors by performing sensitivity 
analysis. This analysis was performed by coupling \Cyclus with Dakota 
and identifying variables to perturb based on the results of Chapters 4 and 
5. The \acrfull{OAT} analysis identified the 
trends from varying each input parameter independently, and how the deployment 
scheme for this work impacts each of the results. The analysis also 
identified tradeoffs between different reactor designs, such as how the Xe-100 
needs more \gls{HALEU} than the VOYGR, but a smaller fuel mass. This 
analysis also identified the transition start time as not impactful on 
the results as the other parameters. The synergistic 
analysis identified how some of the input parameters interact to affect the output 
metrics. Some of the combinations of input parameters (e.g., varying 
the \gls{LWR} lifetime and the VOYGR build share) had trends consistent 
across the input parameter space and consistent with the results of 
the \gls{OAT} analysis. Other combinations (e.g., varying the Xe-100 
burnup and the \gls{MMR} share) varied in their effect across the input parameter 
space because of interactions between the parameters. Finally, the global 
sensitivity analysis quantified the effect of different input parameters 
on the variance of the output metrics. All of the analysis 
identified the Xe-100 discharge burnup as the most impactful parameter
for almost every metric, regardless of which advanced reactor build share 
was varied. This result stems from the strong relationship between 
the Xe-100 burnup and the different metrics, as well as the methodology 
of the deployment scheme. Based on the deployment scheme, the number of 
Xe-100s deployed no matter which advanced reactor is selected for 
build share variation. Therefore, the material needs of the 
Xe-100, which is closely tied to the discharge burnup, always 
affects the metrics if there is variation in the advanced reactor 
build shares. 

% Optimization
Chapter 7 demonstrated a methodology to optimize the once-through 
transitions and identified transitions that minimize different 
fuel cycle metrics. Results from this chapter show that minimizing the 
\gls{SWU} capacity to produce \gls{HALEU} requires maximizing the number of 
VOYGRs built, minimizing the mass of \gls{SNF} requires maximizing the number 
of Xe-100s built, and minimizing both of these fuel cycle metrics is a 
balance between deploying Xe-100s and VOYGRs. The results for each 
of these optimization problems show agreement in maximizing the number 
of \glspl{LWR} operating for 80 years, maximizing the Xe-100 burnup,
and minimizing the \gls{MMR} build share. The \gls{MMR} burnup becomes
irrelevant if no \glspl{MMR} are built. The results from the optimization 
work was consistent with the results of the sensitivity analysis, but 
they did not fully meet expectations. The input parameters identified that 
were not subject to a linear constraint (i.e., not the advanced reactor 
build shares) matched expectations based on the trends of the sensitivity 
analysis and intuition. However, The methodology employed for this 
analysis did not adhere well to the linear constraint of the advanced reactor 
build shares in all three of the problems. Therefore, the 
results from the optimization work can not be taken at face value 
and are better used to identifying a relationship between the advanced 
reactor build shares than a specific set of parameters. Additionally, 
the optimization problems were limited in the number of evaluations
performed, based on limited computational resources. Allowing the 
algorithm to run more evaluations may produce results that better 
meet expectations and the linear constraint. 

% Neutronics
Finally, Chapter 8 described the Xe-100-like and 
\gls{MMR}-like reactor models developed and 
used to analyze the performance of \gls{HALEU} produced from 
downblended \gls{HEU}. The downblended \gls{HEU} compositions 
were compared against the performance of \gls{HALEU} with only 
$^{235}$U and $^{238}$U, referred to as the ``pure'' fuel. 
Performance metrics considered include the 
\keff, \betaEff, energy- and spatially-dependent flux, and 
the fuel, coolant, moderator, and total reactivity temperature 
feedback coefficients. The results of the Xe-100-like 
reactor in an equilibrium state show that the impurities from 
the downblended 
\gls{HEU} fuel increases the \keff but decreases the \betaEff, 
compared with the results of the pure fuel. 
The energy-dependent flux shows a decrease in the thermal flux 
around 0.1 eV and an increase in the fast flux above 20 MeV 
when using the downblended \gls{HEU}. The spatially-dependent 
flux shows a decrease in the thermal (below 0.625 eV) and 
fast (above 0.625 eV) fluxes. The fluxes also showed an asymmetry 
in the axial fluxes for both energy groups that is 
a result of the distribution of pebbles at different burnup levels 
throughout the height of the core. The reactivity temperature 
feedback coefficients from each fuel composition are within 
error of the coefficients from the pure fuel or are consistent 
with published results \cite{mulder_neutronics_2020}.

The results of the \gls{MMR}-like model at \gls{BOL}, \gls{MOL}, 
and \gls{EOL} show that the different \gls{HALEU} compositions 
result in lower \keff values than the pure fuel at each burnup 
step. However, 
the reactor still has a \keff above 1 at the \gls{EOL}, 
which suggests that the change in \keff would not prevent this 
design from reaching its designed 20 year lifetime. The 
different \gls{HALEU} compositions resulted in \betaEff that 
are within the values from the pure fuel. The energy-dependent 
flux showed that the downblended \gls{HEU} mostly affects 
the fast (above 0.625 eV) flux, causing a small increase in 
the flux. The spatially-dependent flux showed some 
asymmetry in the axial direction from one of the downblended 
\gls{HEU} compositions at \gls{MOL}, but no other large 
effects. The reactivity temperature feedback coefficients 
are mostly within error of the coefficients from the pure fuel. 
The coolant reactivity feedback coefficients from the downblended 
\gls{HEU} are outside error from those from the pure fuel, 
but these values have a much smaller magnitude than the other 
reactivity feedback coefficients and operate on a much 
slower time scale. Based on the performance metrics of 
both reactor designs with the different \gls{HALEU} 
compositions, the impurities present in the downblended 
\gls{HEU} may lead to small adjustments in performance, but do 
not lead to large changes in operation or prevent the 
reactors from operating in a safe state. 
 
\section{Limitations and Future Work}
The work performed here provides a foundation for continued analysis and 
exploration of the fuel cycle impacts of deploying \gls{HALEU}-fueled reactors. 
One limitation of this work is that is explores the impacts of 
deploying \gls{HALEU}-fueled reactors at a very macroscopic level. 
Potential future work to address this limitation would model and 
explore the impacts of these reactors on a more microscopic level. 
This includes translating potential demands to facility capacities, 
numbers, and designs. For example, designing and comparing the 
centrifuge cascades required to produce the enriched uranium for 
each of the advanced reactors. This work could account for the 
different \gls{NRC} facility classifications based on the 
enrichment level of the handled material, and explore how to 
most efficiently develop and use different facilities to produce the 
enriched uranium. This analysis would provide more detailed analysis 
of the fuel cycle needs for supporting \gls{HALEU}-fueled reactors. 

This work also focused on the uranium and heavy metals for 
the fuel, but this is not the only component of fuel or reactor 
cores. Another area of future work includes modeling non-fuel 
materials needed to support these transitions, such as the amount 
of reactor-grade graphite 
to be the moderators in the Xe-100 and \gls{MMR}. This analysis would 
provide insight into other supply chain requirements that equally 
impactful on establishing these fuel cycles. The fuel cycle modeling 
methodology employed in this work could provide a foundation 
for how to carry out this analysis. 

A third limitation of this work is the disregard for nonproliferation 
safeguards in the fuel cycle modeling. Nonproliferation safeguards 
are an important part of ensuring the peaceful uses of nuclear 
power, and are implemented across fuel cycle facilities. To 
address this limitation, potential future work would develop 
a method to incorporate safeguards into fuel cycle modeling. 
Incorporating safeguards into fuel cycle modeling could 
restrict facility throughputs, minimize idle \gls{SWU} 
capacity, and limit the amount of material that can be transported 
at a time. Each of these restrictions can affect the ability 
to produce enough fuel for the deployed reactors, and investigation 
of the effects of safeguards on the fuel cycle provides more 
detailed metrics of how to develop and support a fleet of 
\gls{HALEU}-fueled reactors. 

Future work can also expand upon the analysis of the downblended 
\gls{HEU} on the reactor performance. For example, modeling other 
\gls{HALEU}-fueled reactors, such as the Oklo Aurora, that 
have announced an intent to use \gls{HALEU} created from the identified 
\gls{HEU} stockpiles. Additionally, expand 
the analysis to include other metrics, such as power peaking factors and 
power distributions in the core, or kinetics and dynamics parameters. 
Expanding the analysis to include these parameters would provided 
more details on how the impurities from the \gls{HEU} may 
affect reactor operation. One can also increase the 
model fidelity by varying temperatures across the core, 
adding burnable absorbers to the materials, or altering the 
geometry to better match with published information. These 
updates to the models would provide information 
about how the \gls{HALEU} composition would affect the 
reactor performance in a more realistic simulation. 