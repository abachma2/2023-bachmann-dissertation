\begin{figure}
    \centering
    \begin{tikzpicture}[node distance=1.2cm]
        \node (mine) [facility] {Uranium Mine};
        \node (enrichment) [facility, below of=mine]{Enrichment};
        \node (reactor) [facility, below of=enrichment]{Reactor};
        \node (wetstorage) [facility, below of=reactor, text width=1.2cm]{Cooling Pool};
        \node (sinkhlw) [facility, below of=wetstorage, xshift=1.2cm]{Repository};
        \node (separation) [transition, right of=wetstorage,xshift=1cm]{Separations};
        \node (fuelfab) [transition, above of=separation]{Fuel Fab};
        %\node (fr_enrichment) [transition, right of=enrichment,xshift=3cm, text width=1.2cm]{FR Enrichment};
        \node (sfr) [transition, right of=fuelfab, xshift=1.cm,text width=1.2cm]{Fast Reactor};
        
        \draw [arrow] (mine) -- (enrichment);
        \draw [arrow] (enrichment) -- (reactor);
        \draw [arrow] (reactor) -- (wetstorage);
        \draw [arrow] (wetstorage) |- (sinkhlw);
        \draw [arrow] (separation) -- (fuelfab);
        \draw [arrow] (separation) -- (sinkhlw);
        \draw [arrow] (fuelfab) -- (sfr);
        \draw [dashed, ->] (wetstorage) -- (separation);
        \draw [arrow] (sfr) |- (separation);
        \draw [arrow] (enrichment) -| (sfr);

        \end{tikzpicture}
    \caption{Fuel cycle facilities and material flow between facilities for 
    the continuous recycling scenarios.}
    \label{fig:continuous_fuel_cycle}
\end{figure}