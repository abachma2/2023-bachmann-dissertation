\section{Methdology}
Recovery and downblending of \gls{HEU} to produce \gls{HALEU} means that 
the fuel is likely to contain impurities present in the original
\gls{HEU}. Known impurities in potential \gls{HEU}
supplies to create \gls{HALEU} include $^{232}$U and $^{236}$U
\cite{vaden_isotopic_2018,nelson_foreign_2010},  
which are parasitic neutron absorbers and have the potential to affect 
reactor physics and reactor operation. To investigate the magnitude of this 
effect, this step will model the use of 
impure \gls{HALEU} fuel in a reactor and compare it to the use of pure 
\gls{HALEU} (only $^{235}$U and $^{238}$U)
to understand how the impurities affect the performance of the reactor. 

Full core models of the Xe-100 and \gls{MMR} will be created in Serpent 
\cite{leppanen_serpent_2014} to model the neutronics and depletion of 
each reactor 
design. The fresh fuel composition for each core will be varied to provide 
a comparison of pure and impure fuel.
Material compositions of impure fuel will be modeled based on the 
known compositions of \gls{HALEU} from the \gls{EBR} stockpile 
\cite{vaden_isotopic_2018} and the Y-12 stockpile \cite{nelson_foreign_2010}.
Each core will be modeled three times: once modeling pure \gls{HALEU}, 
once modeling impure \gls{HALEU} from the \gls{EBR} stockpile, and once 
modeling impure \gls{HALEU} from the Y-12 stockpile.
By modeling the core composition to be entirely composed of impure fuel 
this work reports the most extreme effect that these impurities can 
have on reactor performance. Understanding the limitations of using only 
impure \gls{HALEU} informs on if downblended \gls{HEU} can be used in 
these reactors exclusively for initial reactor core loadings, or if 
infrastructure for enriching natural uranium must be established to 
produce \gls{HALEU} for initial core loadings. 

Results of each simulation to be compared include the neutron flux at 
\gls{BOL}, mid-cycle, and \gls{EOL}, $k_{eff}$ as a function of burnup, 
fuel temperature reactivity coefficient, and 
$\beta_{eff}$ as a function of burnup. Each parameter provides a 
measurement of the performance of 
the reactor, such as the materials degradation rate, amount of burnable 
poisons required, control rod worth, and the cycle time. Investigating 
each of these results helps to determine if the impurities potentially 
present in \gls{HALEU} will prevent any of the design criteria of the 
reactors from being met. 