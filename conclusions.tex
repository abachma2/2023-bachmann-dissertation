% General Summary


The goal of this work is to investigate the impacts of deploying reactors 
fueled 
by \gls{HALEU} in the United States, including the impacts that the reactors 
have on the \gls{NFC} and the impacts the \gls{NFC} has on the reactors. 
The results of this work are intended to
aid and guide policy makers and key stake holders on how to best establish a 
fuel cycle to support the deployment of \gls{HALEU}-fueled reactors. 
Within this primary goal, there are three specific objectives:
\vspace{0.2cm} 
\noindent
\begin{enumerate}
\item Quantify potential material requirements for the transition 
from \glspl{LWR} to advanced reactors in open and closed 
fuel cycles.

\item Understand the impacts of fuel cycle parameters on the material 
requirements and design optimized transition scenarios.

\item Identify potential limitations in using downblended \gls{HEU} 
on reactor performance.

\end{enumerate}

These goals were met by designing and modeling various transitions 
to advanced reactors, considering a once-through and a closed 
fuel cycle. The once through fuel cycles were then investigated 
further by performing sensitivity analysis and optimization. Finally, 
models of two different \gls{HALEU}-fueled advanced reactors were 
created to investigate the impact of different uranium isotopic 
compositions of the \gls{HALEU} fuel on reactor performance. 

% Once-through analysis

% Closed fuel cycle analysis

% Sensitivity analysis
Chapter 6 examined the effects of different input parameters for 
their effects on different output metrics for the fuel cycle transition from 
\glspl{LWR} to different advanced reactors. The \gls{OAT} analysis identified the 
trends from varying each input parameter independently, and how the deployment 
scheme for this work impacts each of the results. The analysis also 
identified tradeoffs between different reactor designs, such as how the Xe-100 
needs more \gls{HALEU} than the VOYGR, but a smaller fuel mass. The synergistic 
analysis identified how some of the input parameters interact to affect the output 
metrics. While some of the combinations of input parameters (e.g., varying 
the \gls{LWR} lifetime and the VOYGR build share) had consistent trends 
across the input parameter space, some combinations (e.g., varying the Xe-100 
burnup and the \gls{MMR} share) varied in their effect across the input parameter 
space. Finally, the global sensitivity analysis quantified the effect of different 
input parameters on the output metrics. We build surrogate models to fully capture 
the input parameter space. The models were consistent with the Sobol' indices, 
such that the Xe-100 burnup is the most impactful parameter when varying any 
of the three advanced reactor build shares. 

% Optimization
Chapter 7 demonstrated a methodology to optimize the once-through 
transitions and identified transitions that minimize different 
fuel cycle metrics. Results from this chapter show that minimizing the 
\gls{SWU} capacity to produce \gls{HALEU} requires maximizing the number of 
VOYGRs built, minimizing the mass of \gls{SNF} requires maximizing the number 
of Xe-100s built, and minimizing both of these fuel cycle metrics is a 
balance between deploying Xe-100s and VOYGRs. The results from the optimization 
work was consistent with the results of the sensitivity analysis, but 
they did not fully meet expectations. The input parameters identified that 
were not subject to a linear constraint (i.e., not the advanced reactor 
build shares) matched expectations based on the trends of the sensitivity 
analysis and intuition. However, The methodology employed for this 
analysis did not adhere well to the linear constraint of the advanced reactor 
build shares in all three of the problems. Therefore, the 
results from the optimization work can not be taken at face value 
and are better used to identifying a relationship between the advanced 
reactor build shares than a specific set of parameters. 


% Neutronics
 
\section{Future Work}
The work performed here provides a foundation for continued analysis and 
exploration of the fuel cycle impacts of deploying \gls{HALEU}-fueled reactors. 
Potential areas of future work include modeling non-fuel materials needed 
to support these transitions, such as the amount of reactor-grade graphite 
to be the moderators in the Xe-100 and \gls{MMR}. This analysis would 
provide insight into other supply chain requirements that equally 
impactful on establishing these fuel cycles. Additionally, the 
results of this work can be used to determine facility sizes, throughputs, 
and numbers. This information aids in developing the supply chains 
to support these transitions, but also in designing and evaluating 
safeguards for these transitions. Potential safeguards measures may 
limit the facility sizes and the amount of material that can move from 
one facility to another, which affects how the transition can be 
implemented. By 
incorporating facility information into the fuel cycle model, the 
sensitivity analysis and optimization methodologies can be used 
to identify ways to adjust the transition to better account for 
the facility limitations and safeguards needs. 

The analysis of the downblended \gls{HEU} on the reactor performance
can also be expanded. For example, consider other \gls{HALEU}-fueled 
reactors, such as the Oklo Aurora, that have announced an intent to 
use \gls{HALEU} created from these stockpiles. Additionally, expand 
the analysis to other metrics, such as power peaking factors and 
power distributions in the core, or 
increase the fidelity of the models. Ways to increase the model 
fidelities include incorporating burnable poisons, control rods, 
and couple with heat transfer calculations to create a multi-physics model. 
These incorporations into the models will provide greater 
accuracy to the results. 