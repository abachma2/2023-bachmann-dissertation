The United States is considering the deployment of advanced reactors that 
require uranium enriched between 5-20\% $^{235}$U, often referred to 
as \gls{HALEU}. At the present, there are no commercial facilities 
in the US to produce \gls{HALEU}, 
prompting questions of how to create a dependable supply chain of 
\gls{HALEU} to support these reactors. 
\gls{HALEU} can be produced through two primary methods: downblending 
\gls{HEU} and enriching natural uranium. The amount of \gls{HEU} available 
and impurities present in the \gls{HEU} limit downblending capabilities.
The \gls{SWU} capacity and amount of natural uranium available limit 
enriching natural uranium capabilities. To understand the resources necessary 
to commercially produce \gls{HALEU} with each of these methods, one can 
quantify the material requirements of transitioning to \gls{HALEU}-fueled 
reactors. 

In this dissertation, we model the transition from Light Water 
Reactors to different 
advanced reactors, considering once-through and closed 
fuel cycles to determine material requirements for supporting 
these fuel cycles. 
Material requirements of interest across this work include the 
mass of enriched uranium, mass of \gls{HALEU}, feed uranium, 
\gls{SWU} capacity, and the mass of used fuel sent for disposal.
We use \Cyclus and publicly-available 
information about Light Water Reactors, the X-energy Xe-100, the
Ultra Safe Nuclear Corporation Micro Modular Reactor, and the 
NuScale VOYGR to model potential transition scenarios and demonstrate 
the methodologies developed in this work.
To more accurately model the closed fuel 
cycles, we develop a new \Cyclus archetype, called OpenMCyclus, 
that couples with OpenMC to dynamically model fuel depletion in a 
reactor and provide more accurate used fuel compositions.
The results of this transition analysis show how the 
characteristics of the advanced reactors deployed drive 
the materials required to support the fuel cycle. Closing the 
fuel cycle reduces the materials required, but the 
reduction in materials is driven by the amount of material 
available for reprocessing. 

To gain more insight into how transition parameters not 
considered in the transition analysis affect material 
requirements, we perform sensitivity 
analysis on one of the once-through transitions by coupling 
\Cyclus with Dakota. The results of 
the sensitivity analysis highlight some of the trade-offs between 
different reactor designs. One such tradeoff is the increased 
\gls{HALEU} demand but decreased used fuel discharged when increasing 
the Xe-100 deployment and decreasing the VOYGR deployment. Additionally, 
these results identify the Xe-100 discharge burnup as consistently 
being one of the most impactful input parameters for this transition, 
because of how the deployment scheme in this work affects the number of 
Xe-100s built no matter which advanced reactor build share is 
specified. 

To identify potential transitions that minimize material requirements,
we then use the \Cyclus-Dakota to optimize a once-through transition
using the genetic algorithms in Dakota. In single-objective problems to 
minimize the \gls{SWU} capacity required to produce \gls{HALEU} and 
minimize the amount of used nuclear fuel, the algorithm finds 
solutions that are consistent with the results of the 
sensitivity analysis. The results cannot be taken at face 
value, because the algorithm did not fully converge and the genetic 
algorithms do not enforce the applied
linear constraint for the advanced reactor build shares to sum to 
100\%. However, the results provide guidance on how to 
adjust the 
input parameters to optimize the transition for a minimal \gls{HALEU} 
\gls{SWU} or the used fuel mass. Parameter adjustments include 
maximizing the 
number of Light Water Reactors that receive license extensions to 
operate for 80 years. Similar results occur when using 
this method for a multi-objective problem to minimize both the 
\gls{HALEU} \gls{SWU} capacity and the used fuel mass. 

Finally, we use neutronics models of the Xe-100 and Micro Modular Reactor 
reactor designs to 
evaluate the steady-state reactor physics performance of downblended 
\gls{HEU} in these 
two designs. We compare the performance of the downblended 
\gls{HEU} to nominally enriched fuel, based on the 
\keff, \betaEff, energy- and spatially-dependent neutron 
fluxes, as well as the fuel, moderator, coolant, and total reactivity 
temperature feedback coefficients. The differences in the fuel 
compositions leads to differences in each of the metrics. 
However, these differences are within error of the 
results of the nominally enriched fuel, or would not prevent the 
reactor from meeting stated design specifications or operating 
in a safe state. 

The work completed in this dissertation develops and demonstrates a 
methodology for modeling fuel cycle transitions and  
understanding the effects of deploying \gls{HALEU}-fueled reactors 
in the US. The effects investigated in these example scenarios 
include various materials and 
resources required to support these reactors, and how the 
parameters of the transition affect these requirements. The 
information generated from this new methodology can be used 
to develop the necessary infrastructure 
and supply chains for support a transition to \gls{HALEU}-fueled 
reactors. Futhermore, this work explores how the 
\gls{HALEU} production method (enriching compared with downblending)
affects reactor performance. 
