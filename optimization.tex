\section{Methodology}
Optimization of the nuclear fuel cycle is intended to develop 
a fuel cycle based on a specific objective or objectives. For this 
work, we developed three different optimization problems to apply to 
Scenarios 7 (no growth, once-through transition to the \gls{MMR}, Xe-100, 
and VOYGR) and 14 (no growth, limited recycling transition to the 
\gls{MMR}, Xe-100, and VOYGR): minimizing the \gls{SWU} capacity to 
produce \gls{HALEU}, minimizing the mass of \gls{SNF} for disposal, 
and minimizing both the \gls{HALEU} \gls{SWU} and the \gls{SNF} 
mass. To optimize these transitions, we consider six different 
variables: percent of \glspl{LWR} operating for 80 years, the build share 
of Xe-100s, the build share of \glspl{MMR}, the build share of VOYGRs, 
the discharge burnup of the Xe-100, and the discharge burnup of the 
\gls{MMR}. All of these variables were considered in the sensitivity 
analysis. The transition start time was also considered in the sensitivity 
analysis. However, that analysis showed that the transition start time has
a much smaller impact on each of the metrics and delaying the transition 
can lead to unfulfilled energy demand. Therefore, this parameter is 
not considered in the optimization of each scenario. 

For this work, the percent of \glspl{LWR} operating for 80 years 
was constrained to between 0-50\% of the current \gls{LWR} fleet. 
The build share of each advanced reactor was allowed to range between 
0-100\%, but the three parameters had to sum to 100\%. The discharge burnups 
are restricted to the values considered in the sensitivity analysis.

We coupled \Cyclus \cite{huff_fundamental_2016} with Dakota 
\cite{adams_dakota_2019} to perform this optimization. For the 
single-objective problems, we used the ``soga'' solve method in 
Dakota, which is a single-objective genetic algorithm. For the 
multi-objective problems we used the ``moga'' solve method in 
Dakota, which is a multi-objective genetic algorithm. Genetic algorithms
have previously been used to optimize fuel cycle transitions 
\cite{passerini_systematic_2014}, which led to the decision to use 
them for this work.

\section{Once-through optimization}

\section{Recycling optimization}



\cite{andrianov_optimization_2019} minimized total discounted costs using 
the MESSAGE energy planning software. 

I get a general sense that a lot of work has been done to optimize fuel cycles 
for cost, unf management, or proliferation risk. Not seeing a whole lot on 
material requirements. 
Methods I'm seeing: genetic algorithms, linear programming, stochastic linear 
programming (same thing?)
