\begin{frame}
      \frametitle{Conclusions}
      \begin{itemize}
        \item This work investigates the impacts of deploying \gls{HALEU}-fueled 
              reactors on the nuclear fuel cycle.
        \item The material requirements of the transitions modeled are governed 
              by the enrichment level and fuel utilization of the advanced 
              reactors deployed
            \begin{itemize}
              \item Deploying only \glspl{MMR} has the largest feed uranium and
                    \gls{SWU} requirements
              \item Deploying primarily VOYGRs has the largest enriched uranium 
                    requriements
            \end{itemize}
        \item More work is needed to complete this Investigation
        \begin{itemize}
            \item Model the transition to advanced reactors assuming a recycle fuel cycle
            \item Perform sensitivity analysis to investigate and quantify the effect 
                  of model parameters on the material requirements
            \item Identify optimized transition scenarios based on minimizng material 
                  requirements
            \item Investigate any potential limits in using downblended \gls{HEU} 
                  in advanced reactors
        \end{itemize}
      \end{itemize}
\end{frame}

\begin{frame}
      \frametitle{Limitations and Future Work}
      \begin{itemize}
            \item Transition analysis provided a macroscopic view of 
                  material needs, focused on fuels
            \begin{itemize}
                  \item<2-> Model the needs of non-fuel materials, 
                        like reactor-grade graphite
                  \item<2-> Determine and model facility capacities
                  \item<2-> Account for processing time 
            \end{itemize}
            \item<3-> Ignores other externalities, like nonproliferation
                    safeguards
            \begin{itemize}
                  \item<4-> Incorporate potential safeguards measures into models
            \end{itemize}
      \end{itemize} 
\end{frame}